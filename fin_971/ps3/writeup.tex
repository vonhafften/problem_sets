\documentclass{article}
\usepackage{amsmath,amsthm,amssymb,amsfonts}
\usepackage{setspace,enumitem}
\usepackage{natbib}
\usepackage{afterpage}
\usepackage{booktabs}
\usepackage{pdfpages}
\usepackage{geometry}
\usepackage{graphicx}
\geometry{margin = 1in}
\graphicspath{ {./adjustment/figures/} }

\setlength{\parindent}{0cm}

\newcommand{\R}{\mathbb{R}}


\begin{document}

\section*{FIN 971: Problem Set 3\footnote{Instructor: Dean Corbae}}

Alex von Hafften

\today

\bigskip

\textbf{Exercise 3.15 of Tirole (project riskiness and credit rationing).} 

\bigskip

Consider the basic, fixed-investment model covered in Section 3.2 of Tirole (2006). In particular, investment is a fixed size $I$, the entrepreneur borrows $I-A$, the probability of success is either $p_H$ (which yields no private benefit) or $p_L$ (which yields private benefit $B$), success yields verifiable revenue $R$ while failure yields 0. There are two types, ``A" and ``B", of the projects, which differ only with respect to ``riskiness" defined by $p^A_HR^A = p^B_HR^B$, but $p^A_H > p^B_H$ so that project $B$ is ``riskier".  The investment cost $I$ is the same for both variants and furthermore, $\Delta p = p^A_H - p^A_L= p^A_H - p^A_L$. Which type of project is less prone to credit rationing?

\bigskip

\textbf{Solution:} Following the logic outlined in Section 3.2 of Tirole (2006), we can derive the $\bar{A}^i$ for each project $i$.  First, assume that both projects have positive NPV if the entrepreneur the behaves (i.e. $p_H^i R^i > I$) and negative NPV if the entrepreneur does not behave (i.e., $p_L^i R^i + B < I$), which means that the lending contract will require the entrepreneur to behave.  The borrower incentive compatibility constraint to behave is 

$$
p_H^i R_b^i \ge p_L^i R_b^i + B \iff (\Delta p) R_b^i \ge B
$$

Thus, the most that can be pledged to the lender without violating the borrower IC constraint is

$$
R_l^i = R -\frac{B}{\Delta p}
$$

The participation constraint of the lender is

$$
p_H^i R_l^i \ge I - A \iff p_H^i \Bigg( R -\frac{B}{\Delta p} \Bigg) \ge I - A
$$

Where the borrower IC holds with equality. Thus, the net worth level at which financing is possible is:

$$
\bar{A}^i = p^i_H \frac{B}{\Delta p } - (p^i_H R^i - I)
$$

Entrepreneurs with net worth $A \ge \bar{A}^i$ get funded; otherwise, not.  Thus, the project that is less prone to credit rationing is the project with the lower $\bar{A}^i$:

$$
\bar{A}^A > \bar{A}^B 
\iff 
p^A_H \frac{B}{\Delta p } - (p^A_H R^A - I) > p^B_H \frac{B}{\Delta p } - (p^B_H R^B - I)
\iff
p^A_H  > p^B_H 
$$

Thus, project B is less prone to credit rationing.

\pagebreak

\textbf{Exercise 3.13 of Tirole (lender market power with fixed investment).} 

\bigskip

The environment is similar to Section 3.2 of Tirole with one exception. An enterpreneur has internal wealth $A$ (which could be negative because of previous debt) and wants to undertake non-negative investment $I > A$ into a fixed size project. The project yields $R > 0$ with probability $p$ and 0 with probability $1 - p$. The probability of success is $p_H$ if the entrepreneur works and $p_L < p_H$ if he shirks. The entrepreneur obtains private benefit $B$ if she shirks and 0 otherwise. The borrower is protected by limited liability and everyone is risk neutral. The project is worthwhile only if the entrepreneur behaves.

\bigskip

The exception is that there is a single lender. This lender has access to funds that command an expected rate of return equal to 0 (so the lender would content himself with a 0 rate of return, but will use his market power to obtain a superior rate of return). Assume $V \equiv p_HR - I > 0$ and let $\bar{A}$ and $\hat{A}$ be defined by

\begin{align*}
\bar{A} &\equiv I - p_H \Bigg[R - \frac{B}{\Delta p}\Bigg] \\
\hat{A} &\equiv p_H \frac{B}{\Delta p}
\end{align*}

where $\Delta p = p_H - p_L$. Assume that $\bar{A} > 0$ and that the lender makes a take-it-or-leave-it offer to the borrower (i.e. the lender chooses $R_b$, the borrower's compensation in the case of success).

\bigskip

(i) What contract is optimal for the lender? Be sure to state the programming problem explicitly.

\bigskip

The lender maximizes their profit subject to some constraints. First, we require that the borrower behaves, so their IC constraint is that expected payoff from behaving $p_HR_b$ exceeds the expected payoff from not behaving $p_L R_l + B$.  Subject to this IC, the lender's expected profit is the expected return on the loan (given the behaving borrower) $p_H R_l$ minus the initial loan size $I - A$. We also should consider the participation constraints of the lender and the borrower.  The borrower could eat their net worth $A$ (if $A > 0$) instead of their expected payoff $p_H R_b$ (given the borrower behaves).  The lender could eat the loan amount $I - A$ instead of their expected payoff $p_H R_l$  (given the borrower behaves).  Finally, by definition $R_l + R_b = R$.  Thus, the programming problem is:

\begin{align*}
\max_{R_l, R_b} & p_H R_l - I + A & \\
\text{s.t. }  
p_H R_b &\ge p_L R_b + B & [IC_b] \\
p_H R_b &\ge A & [PC_b] \\
p_H R_l &\ge I - A & [PC_l] \\
R_l + R_b &= R & \\
\implies 
\max_{R_b} & p_H (R - R_b) - I + A & \\
\text{s.t. }  
R_b &\ge \frac{B}{\Delta p} & [IC_b] \\
R_b &\ge \frac{A}{p_H} & [PC_b] \\
R_b &\le R - \frac{I - A}{p_H} & [PC_l]
\end{align*}

Thus, the lender will choose the lowest $R_b$ such that all constraints hold.  First let $\hat{A}$ be the net worth where $IC_b$ and $PC_b$ both bind:

$$
\frac{B}{\Delta p} = \frac{\hat{A}}{p_H} 
\implies
\hat{A} = p_H \frac{B}{\Delta p}
$$

Second, observe that the $PC_b$ and $PC_l$ cannot both bind.  Suppose not then

$$
\frac{A}{p_H} = R - \frac{I - A}{p_H} \implies 0 = p_H R - I = V > 0 \Rightarrow \Leftarrow
$$

Third, let $\bar{A}$ be the net worth where the $IC_b$ and $PC_l$ both bind:

$$
\frac{B}{\Delta p} = R - \frac{I - \bar{A}}{p_H} 
\implies 
\bar{A} = I - p_H \Bigg(R + \frac{B}{\Delta p} \Bigg)  
$$

Thus, the optimal lending contract for the lender depends on $A$.  If $A < \bar{A}$, then the lenders participation constraint does not hold, so there's no contract (i.e. credit rationing).  If $\bar{A} \le A < \hat{A}$, the borrowers incentive compatibility constraint binds and the borrowers participation constraint is slack, so $R_b = \frac{B}{\Delta p}$.  At $A = \hat{A}$, both the borrowers incentive compatibility and participation constraints bind. If $I > A > \hat{A}$, the borrowers incentive compatibility constraint is slack and the borrowers participation constraint binds, so $R_b = \frac{A}{p_H}$.

\bigskip

(ii) Is the financing decision affected by lender market power (i.e. compared to the case of competitive lenders in Section 3.2)?

\bigskip

The lender market power does not affect the credit rationing threshold. With or without market power, financing does not happen if $A < \bar{A}$.  Lender market power affects the returns; it allows the lender to extract more surplus from the project instead of the borrower extracting the surplus, as was the case with competitive lenders.

\bigskip

(iii) Draw the borrower's net utility (i.e. net of $A$) as a function of $A$. Note that unlike the monotonic case in Section 3.2, it is nonmonotonic among the regions $(-\infty, \bar{A}), [\bar{A}, \hat{A}), [\hat{A}, I)$. Explain.

\bigskip

If $A < \bar{A}$, the lender participation constraint is not satisfied, so there's no lending and the borrower eats $A$. Net of $A$, the borrowers utility is zero.\footnote{Here, I assume that consumption can be negative.  If not, then if $A < 0 < \bar{A}$, lending does not happen and the borrower consumes nothing, so net of $A$, their utility is $-A$.  I suppose an equivalent alternative assumption is that consumption must be nonnegative and the borrower has some other baseline level of consumption $C > -A$, which we're also netting out.}  If $\bar{A} \le A < \hat{A}$, lending happens at $R_b = B/\Delta p$. Thus, the borrowers utility net of $A$ is $p_HB/\Delta p - A$.  If $I >A \ge \hat{A}$, then lending happens at $R_b = A/p_H$, so the borrower utility net of $A$ is $p_HA/p_H - A = 0$.

\pagebreak

\textbf{Exercise 3.5 of Tirole (continuous investment and decreasing returns to scale).} 

\bigskip

Consider the continuous investment model of Section 3.4 of Tirole (2006) with one modification; investment $I$ yields return $R(I)$ in the case of success and 0 in the case of failure, where $R' > 0$ and $R'' < 0, R'(0) > 1/p_H, R'(\infty) < 1/p_H$. The rest of the model is unchanged. That is, the entrepreneur starts with cash $A$, the probability of success is either $p_H$ if he behaves or $p_L$ if he misbehaves. The entrepreneur obtains private benefit $BI$ if he misbehaves and 0 otherwise. Only the final outcome is observable. Let $I^*$ denote the level of investment that maximizes total surplus (i.e. $p_H R'(I^*) = 1$).

\bigskip

(i) How does investment $I(A)$ vary with the level of cash?

\bigskip

...

\bigskip

(ii) How does the shadow value $v$ of cash (the derivative of the borrower's gross utility with respect to cash) vary with the level of cash?


\bigskip

...

\bigskip


\end{document}



