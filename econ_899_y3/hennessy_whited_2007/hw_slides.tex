\documentclass[usenames,dvipsnames, handout]{beamer}
\usetheme{Boadilla}
\usepackage{graphicx}
\usepackage{soul}
\usepackage{multirow}
\usepackage{multicol}
\usepackage{csvsimple}


\newcommand{\R}{\mathbb{R}}
\newcommand{\ubar}[1]{\underaccent{\bar}{#1}}
\newcommand{\Int}{\text{Int}}
\newcommand{\xbf}{\mathbf{x}}
\newcommand{\Abf}{\mathbf{A}}
\newcommand{\Bbf}{\mathbf{B}}
\newcommand{\Gbf}{\mathbf{G}}
\newcommand{\bbf}{\mathbf{b}}
\newcommand{\Lfn}{\mathcal{L}}
\newcommand{\one}{\mathbbm{1}}

\title[HW (2007)]{``How Costly is External Financing? Evidence from a Structural Estimation"\\Christopher A. Hennessy and Toni M. Whited (2007)\\Journal of Finance}
\author{Alex von Hafften}
\institute{UW-Madison}

\begin{document}

\begin{frame}
\titlepage
\end{frame}

\section{Introduction}

\begin{frame}
\frametitle{Motivation}
\small
\begin{itemize}
\item Modigliani-Miller Irrelevance Theorem (1958, 1963) 

\bigskip

\textit{In frictionless world, financing decisions like capital structure (debt vs. equity), payout policy, cash holding, etc. do not matter.}
\bigskip
\item Why? No arbitrage
\bigskip
\item MM assumes there are no financial friction:
\begin{itemize}
\item Perfect and complete capital markets
\item No taxes
\item Bankruptcy is not costly
\item Capital structure does not affect investment policy or cash flows
\item Symmetric information
\end{itemize}
\end{itemize}
\end{frame}



\begin{frame}
\frametitle{Hennessy and Whited (2007)}
\small
\begin{itemize}
\item HW (2007) formulate a dynamic structural model of optimal financial and investment policy for a firm facing
\begin{itemize}
\item Corporate and personal taxes
\item Bankruptcy costs
\item Costs to issue external equity
\end{itemize}
\bigskip
\item HW (2007) estimate parameters describing production technology and financial frictions using simulated method of moments (SMM)
\end{itemize}
\end{frame}


\section{Model}

\begin{frame}
\frametitle{Environment - Production and Debt}
\small
\begin{itemize}
\item Estimated parameters in {\color{red}red}
\item Firm produces with $k$ capital
\item Productivity follows discretized AR(1) process in logs: 

$$
\ln z' = {\color{red}\rho} \ln z + {\color{red}\sigma_\varepsilon} \varepsilon
$$

where $\varepsilon \sim N(0, 1)$. Tauchen discretization $\implies Q(z, z')$ transition probability and finite min/max
\item Operating profits are $z k^{\color{red}\alpha}$ where ${\color{red}\alpha} \in (0, 1)$
\item Firm also has $b$ net debt
\begin{itemize}
\item $b > 0$ is one-period defaultable debt with interest rate $\tilde r$ that depends on $k$, $b$, and $z$ (not contingent on $z'$)
\item $b \le 0$ is cash that returns risk-free rate $r$
\end{itemize}

\item Firm defaults on debt if continuation value is negative
\end{itemize}
\end{frame}

\begin{frame}
\frametitle{Environment - Taxes and Equity Issuance}
\small
\begin{itemize}
\item Personal tax rate $\tau_i \implies$ firms discounts using $\frac{1}{1 + r(1-\tau_i)}$
\item Corporate taxable income is operating profits net of depreciation and interest:
$$
y \equiv z k^\alpha - \delta k - \tilde r(k, b, z^-) b
$$
\item Corporate tax schedule has ``kink" around zero
$$
T^C(x) \equiv 
\begin{cases} 
\tau_c^+ x, & \text{if }x > 0 \\
\tau_c^- x, & \text{if }x \le 0
\end{cases}
$$
\item Shareholder tax liability on dividend:
$$
T^d(X) = \int_0^X \tau_d(x) dx \text{ where } \tau_d(x) \equiv \bar \tau_d * [1 - e^{-{\color{red}\phi} x}]
$$
\item Firm pays fixed, linear, and quadratic costs for external equity issuance:
$$
\Lambda(x) \equiv 
\begin{cases} 
{\color{red}\lambda_0} + {\color{red}\lambda_1} x + {\color{red}\lambda_2} x^2, & \text{if }x > 0 \\
0, & \text{if }x \le 0
\end{cases}
$$
\end{itemize}
\end{frame}



\begin{frame}
\frametitle{``Naive" Way to Write Firm Value Function}
\scriptsize
\begin{align*}
V(k, b, z, z^-) =
\max_{(k', b')} \Bigg\{ 
& \underbrace{w + b' - k'}_{\text{cash dividend } (+) \text{ or equity issuance } (-)} - \underbrace{T^d(w + b' - k')}_{\text{taxes on cash dividend}} - \underbrace{\Lambda(-(w + b' - k'))}_{\text{equity issuance cost}} \\
&+ \frac{1}{1+r(1-\tau_i)} E\Big[\underbrace{\max\{ V(k', b', z', z), 0\}}_{\text{if }V(\cdot) \text{ is } (-) \implies\text{ default}}\Big] \Bigg\} \\
\text{where }
\underbrace{y}_{\text{taxable corporate income}} &\equiv \underbrace{z k^\alpha}_{\text{operating profits}} - \underbrace{\delta k}_{\text{depreciation}} - \underbrace{\tilde r(k, b, z^-) b}_{\text{interest on debt}} \\
\underbrace{w}_{\text{realized net worth}} &\equiv \underbrace{y - T^C(y)}_{\text{after-tax corporate income}}+ \underbrace{k}_{\text{capital}}  - \underbrace{b}_{\text{debt principal}} \\
\underbrace{T^C(x)}_{\text{corporate income tax bill}} &\equiv 
\begin{cases} 
\tau_c^+ x, & \text{if }x > 0 \\
\tau_c^- x, & \text{if }x \le 0
\end{cases}\\
\underbrace{T^d(x)}_{\text{taxes on cash dividend}} &\equiv 
\begin{cases}
\frac{\bar \tau_d}{{\color{red}\phi}}({\color{red}\phi} x + e^{-{\color{red}\phi} x} - 1), & \text{if }x > 0 \\
0,& \text{if } x \le 0
\end{cases} \\
\underbrace{\Lambda(x)}_{\text{equity issuance cost}} &\equiv 
\begin{cases} 
{\color{red}\lambda_0} + {\color{red}\lambda_1} x + {\color{red}\lambda_2} x^2, & \text{if }x > 0 \\
0, & \text{if }x \le 0
\end{cases}
\end{align*}
\end{frame}


\begin{frame}
\frametitle{Smarter Way to Write Firm Value Function}
\scriptsize
\begin{align*}
V(w, z) =
\max_{(k', b')} \Bigg\{ 
& \underbrace{w + b' - k'}_{\text{cash dividend } (+) \text{ or equity issuance } (-)} - \underbrace{T^d(w + b' - k')}_{\text{taxes on cash dividend}} - \underbrace{\Lambda(-(w + b' - k'))}_{\text{equity issuance cost}} \\
&+ \frac{1}{1+r(1-\tau_i)} E\Big[\underbrace{\max\{ V(w', z'), 0\}}_{\text{if }V \text{ is } (-) \text{ can default}}\Big] \Bigg\} \\
\text{where }
\underbrace{y'}_{\text{taxable corporate income}} &\equiv \underbrace{z' (k')^\alpha}_{\text{operating profits}} - \underbrace{\delta k'}_{\text{depreciation}} - \underbrace{\tilde r(k', b', z) b'}_{\text{interest on debt}} \\
\underbrace{w'}_{\text{realized net worth}} &\equiv \underbrace{y' - T^C(y')}_{\text{after-tax corporate income}}+ \underbrace{k'}_{\text{capital}}  - \underbrace{b'}_{\text{debt principal}} \end{align*}
\end{frame}


\begin{frame}
\frametitle{Default and Interest Rates}
\small
\begin{itemize}
\item Firm defaults on debt if $w$ below $z'$-specific threshold:
$$
\underline{w}(z') = V^{-1}(0, z') < 0 \implies z_d(k', b', z) \text{ threshold}
$$
\item If firm defaults, outside investor gets recovery value
\begin{align*}
R(k', z') &= \underbrace{(1-{\color{red}\xi}) (1-\delta)k'}_{\text{depreciated capital net of deadweight bankruptcy cost}} + \underbrace{z'(k')^\alpha}_{\text{operating profit}} \\&- \underbrace{T_c(z'(k')^\alpha - \delta k')}_{\text{corporate tax bill }} - \underbrace{\underline w (z')}_{\text{going-concern}}
\end{align*}
\item Interest rates on debt determined by zero-profit condition for outside investor
\begin{align*}
\underbrace{(1+r(1-\tau_i))b'}_{\text{risk-free investment}} 
&=
\underbrace{(1+(1-\tau_i)\tilde r(k', b', z))b' \int_{z_d(k', b', z)}^{\bar z} Q(z,dz')}_{\text{return on non-defaulted debt}} \\
&+ \underbrace{\int_{\underline z}^{z_d (k', b', z)} R(k', z') Q(z, dz')}_{\text{return on defaulted debt}}
\end{align*}
\end{itemize}
\end{frame}


\begin{frame}
\frametitle{Computation}
\small
\begin{itemize}
\item No closed form solution $\implies$ solve numerically
\bigskip
\item Computational strategy:
\begin{itemize}
\item Guess $\tilde r (k', b', z) = r$
\item Solve $V$ with value function iteration
\item Compute $z_d(k', b', z)$
\item Update $\tilde r(k', b', z)$ using zero-profit condition
\item Repeat until convergence
\end{itemize}
\end{itemize}
\end{frame}



\begin{frame}
\frametitle{Computation Issues}
\small
\begin{enumerate}
\item Discount bond prices are bounded whereas interest rates are not:
\scriptsize
\begin{align*}
V(w, z) =
\max_{(k', b')} \Bigg\{ 
& \underbrace{w + b'q(k',b',z) - k'}_{\text{dividend if } (+) \text{ or equity issuance if } (-)} - \underbrace{T^d(w + b'q(k',b',z) - k')}_{\text{taxes on dividend}} \\
- &\underbrace{\Lambda(-(w + b'q(k',b',z) - k'))}_{\text{equity issuance cost}} + \frac{1}{1+r(1-\tau_i)} E\Big[\underbrace{\max\{ V(w', z'), 0\}}_{\text{if }V \text{ is } (-) \text{, default}}| z\Big] \Bigg\} \\
\text{where }
\underbrace{y'}_{\text{taxable income}} &\equiv \underbrace{z' (k')^\alpha}_{\text{operating profits}} - \underbrace{\delta k'}_{\text{depreciation}} - \underbrace{(1-q(k',b',z)) b'}_{\text{interest on debt}} \\
\underbrace{w'}_{\text{realized net worth}} &\equiv \underbrace{y' - T^C(y')}_{\text{after-tax corporate income}}+ \underbrace{k'}_{\text{capital}}  - \underbrace{q(k',b',z)b'}_{\text{debt principal}}
\end{align*}
\small
\item What is the $w$ grid?
\begin{itemize}
\item HW (2007) are specific about $z$, $b$, $k$ grids, but quiet about the $w$ grid
\item My solution: Loop over $z$, $b$, $k$ grids and solve $w$ for $q = 0$ and $q = 1/(1+r)$, then linear interpolate between min and max
\end{itemize}
\item No contraction mapping for bond prices $\implies$ update $q$ guess slowly
\end{enumerate}
\end{frame}

\section{Estimation}

\begin{frame}
\frametitle{External Parameters from Related Literature}
\small
\begin{center}
\begin{tabular}{ cc } 
 \hline
 Parameter    & Value \\ 
 \hline
 $\bar\tau_d$ & 0.12 \\ 
 $\tau_i$     & 0.29 \\ 
 $\tau_c^+$   & 0.40 \\ 
 $\tau_c^-$   & 0.20 \\ 
 $r$          & 0.025 \\
 $\delta$     & 0.15 \\
 \hline 
\end{tabular}
\end{center}
\end{frame}


\begin{frame}
\frametitle{Moments Selection}
\small
\begin{itemize}
\item Production function curvature $\alpha$
\begin{itemize}
\item Hello
\end{itemize}
\item Fixed, linear, and quadratic external equity issuance costs $\lambda_0, \lambda_1, \lambda_2$
\begin{itemize}
\item Hello
\end{itemize}
\item Bankruptcy deadweight cost $\xi$
\begin{itemize}
\item Hello
\end{itemize}
\item Shareholder tax liability curvature $\phi$
\begin{itemize}
\item Hello
\end{itemize}
\item Productivity shock variance and productivity persistence  $\sigma_\epsilon, \rho$
\begin{itemize}
\item Hello
\end{itemize}
\end{itemize}
\end{frame}

\begin{frame}
\frametitle{Moments and Estimated Parameters}
\scriptsize
\begin{center}
\begin{tabular}{ ccc } 
 \hline
 Parameter            & Estimates (HW) & SE (HW) \\ 
 \hline
 $\alpha$             & 0.627 &(0.219) \\ 
 $\lambda_0$          & 0.598 &(0.233) \\ 
 $\lambda_1$          & 0.091 &(0.026) \\ 
 $\lambda_2$          & 0.0004 &(0.0008) \\ 
 $\xi$                & 0.104 &(0.059) \\
 $\phi$               & 0.732 &(0.844) \\
 $\sigma_\varepsilon$ & 0.118 &(0.042) \\
 $\rho$               & 0.684 &(0.349) \\
 \hline 
\end{tabular}
\end{center}

\begin{center}
\csvautotabular{table_1.csv}
\end{center}
\end{frame}






\end{document}

