\documentclass{article}
\usepackage{amsmath,amsthm,amssymb,amsfonts}
\usepackage{setspace,enumitem}
\usepackage{graphicx}
\usepackage{hyperref}
\usepackage{natbib}
\usepackage{afterpage}
\usepackage{xcolor}
\usepackage{etoolbox}
\usepackage{bbm}
\usepackage{booktabs}
\usepackage{pdfpages}
\usepackage{multicol}
\usepackage{soul}
\usepackage{geometry}
\usepackage{accents}
\usepackage{accents}
\hypersetup{
	colorlinks,
	linkcolor={blue!90!black},
	citecolor={red!90!black},
	urlcolor={blue!90!black}
}
\usepackage{setspace}

\newtheorem{theorem}{Theorem}
\newtheorem{assumption}{Assumption}
\newtheorem{definition}{Definition}
\newtheorem{proposition}{Proposition}
\newtheorem{lemma}{Lemma}
\newcommand{\R}{\mathbb{R}}
\newcommand{\N}{\mathbb{N}}
\newcommand{\Lfn}{\mathcal{L}}
\newcommand{\one}{\mathbbm{1}}
\newcommand{\Int}{\text{Int}}
\newcommand{\ubar}[1]{\underaccent{\bar}{#1}}
\newcommand{\xbf}{\mathbf{x}}
\newcommand{\Abf}{\mathbf{A}}
\newcommand{\Bbf}{\mathbf{B}}
\newcommand{\Gbf}{\mathbf{G}}
\newcommand{\bbf}{\mathbf{b}}

\title{Notes on Hennessy and Whited (2007)}
\author{Alex von Hafften}
\date{\today}

\doublespacing

\begin{document}

\maketitle

\begin{itemize}

\item Notation is a bit confusing in the original paper, so trying to clear up my confusion here.

\item Notation: Current productivity is $z$, yesterday productivity is $z^-$, and tomorrow productivity is $z'$

\item Cash flow from operations:

$$
z \pi(k) \equiv z k ^ \alpha
$$

\item Corporate tax income bill:

$$
T^C(k, b, z^-, z) \equiv [\tau_c^+ \chi + \tau_c^- ( 1 - \chi)] \cdot [z \pi (k) - \delta k - r(k, b, z^-) b]
$$

where  $\chi \equiv \one[z \pi (k) - \delta k - r(k, b, z^-) b > 0]$

\item Individual tax rate $\tau_i$

\item Cash distribution taxes

\begin{align*}
T^d(X) &\equiv \int_0^X \tau_d(x) dx \\
\text{where } 
\tau_d(x) &\equiv \bar \tau_d \times [1 - e^{-\phi x}] \\
\implies
T^d(X) &\equiv 
\begin{cases}
0,& X \le 0\\
\frac{\bar \tau_d}{\phi}(\phi X + e^{-\phi X} - 1), & X > 0
\end{cases}
\end{align*}

\item Costly external equity financing

\begin{align*}
\Lambda(x) = \lambda_0 +\lambda_1 x + \lambda_2 x^2
\end{align*}

where $\lambda_0 \ge 0$, $\lambda_1 \ge 0$, $\lambda_2 \ge 0$

\item State variable is net worth $w$ and choice variables are debt and capital for next period

\item Equity value function

\begin{align*}
V(w, z) 
= \max_{k', b'} \Bigg\{ &\; \; \;  \;  \; 
\underbrace{\Phi [w + b' - k' - T^d(w + b' - k')]}_{\text{cash payment to equity holders}}\\
&\underbrace{- (1-\Phi) [k' - w - b' + \Lambda(k'-w - b')]}_{\text{equity issuance}}\\
&+ \Bigg[ \frac{1}{1 + r ( 1-\tau_i)}\Bigg] E\Bigg[\Big(V(w(k', b', z, z'), z')\Big)^+\Bigg| z \Bigg]\Bigg\}
\end{align*}

where 

\begin{align*}
\Phi &\equiv \one (w + b' > k')\\
w(k', b', z, z') &\equiv z' \pi(k') + (1-\delta) k' - T^C(k', b', z, z') - (1 +r(k', b', z)) b'
\end{align*}

\pagebreak

\item Naive value function:

\begin{align*}
V(k, b, z, z^-) =
\max_{(k', b')} \Bigg\{ 
& \underbrace{w + b' - k'}_{\text{cash dividend if } (+) \text{ or equity issuance if } (-)} \\ 
&- \underbrace{T^d(w + b' - k')}_{\text{taxes on cash dividend}} \\
&- \underbrace{\Lambda(-(w + b' - k'))}_{\text{equity issuance cost}} \\
&+ \frac{1}{1+r(1-\tau_i)} E\Big[\underbrace{\max\{ V(k', b', z', z), 0\}}_{\text{if }V \text{ is } (-) \text{ can default}}\Big] \Bigg\} \\
\text{where }
\underbrace{y}_{\text{taxable corporate income}} &\equiv \underbrace{z k^\alpha}_{\text{operating profits}} - \underbrace{\delta k}_{\text{depreciation}} - \underbrace{\tilde r(k, b, z^-) b}_{\text{interest on debt}} \\
\underbrace{T^C(x)}_{\text{corporate income tax bill}} &\equiv 
\begin{cases} 
\tau_c^+ x, & \text{if }x > 0 \\
\tau_c^- x, & \text{if }x \le 0
\end{cases}\\
\underbrace{w}_{\text{realized net worth}} &\equiv \underbrace{y - T^C(y)}_{\text{after-tax corporate income}}+ \underbrace{k}_{\text{capital}}  - \underbrace{b}_{\text{debt principal}} \\
\underbrace{T^d(x)}_{\text{taxes on cash dividend}} &= 
\begin{cases}
\frac{\bar \tau_d}{\phi}(\phi x + e^{-\phi x} - 1), & x > 0 \\
0,& x \le 0
\end{cases} \\
\underbrace{\Lambda(x)}_{\text{equity issuance cost}} &= 
\begin{cases} 
\lambda_0 + \lambda_1 x + \lambda_2 x^2, & \text{if }x > 0 \\
0, & \text{if }x \le 0
\end{cases}
\end{align*}

\pagebreak

\item Smarter value function:

\begin{align*}
V(w, z) =
\max_{(k', b')} \Bigg\{ 
& \underbrace{w + b' - k'}_{\text{cash dividend if } (+) \text{ or equity issuance if } (-)} \\ 
&- \underbrace{T^d(w + b' - k')}_{\text{taxes on cash dividend}} \\
&- \underbrace{\Lambda(-(w + b' - k'))}_{\text{equity issuance cost}} \\
&+ \frac{1}{1+r(1-\tau_i)} E\Big[\underbrace{\max\{ V(w', z'), 0\}}_{\text{if }V \text{ is } (-) \text{ can default}}\Big] \Bigg\} \\
\text{where }
\underbrace{y'}_{\text{taxable corporate income}} &\equiv \underbrace{z' (k')^\alpha}_{\text{operating profits}} - \underbrace{\delta k'}_{\text{depreciation}} - \underbrace{\tilde r(k', b', z) b'}_{\text{interest on debt}} \\
\underbrace{T^C(x)}_{\text{corporate income tax bill}} &\equiv 
\begin{cases} 
\tau_c^+ x, & \text{if }x > 0 \\
\tau_c^- x, & \text{if }x \le 0
\end{cases}\\
\underbrace{w'}_{\text{realized net worth}} &\equiv \underbrace{y' - T^C(y')}_{\text{after-tax corporate income}}+ \underbrace{k'}_{\text{capital}}  - \underbrace{b'}_{\text{debt principal}} \\
\underbrace{T^d(x)}_{\text{taxes on cash dividend}} &= 
\begin{cases}
\frac{\bar \tau_d}{\phi}(\phi x + e^{-\phi x} - 1), & x > 0 \\
0,& x \le 0
\end{cases} \\
\underbrace{\Lambda(x)}_{\text{equity issuance cost}} &= 
\begin{cases} 
\lambda_0 + \lambda_1 x + \lambda_2 x^2, & \text{if }x > 0 \\
0, & \text{if }x \le 0
\end{cases}
\end{align*}

\pagebreak

\item Smarter value function with bond prices instead of interest rates:

\begin{align*}
V(w, z) =
\max_{(k', b')} \Bigg\{ 
& \underbrace{w + b'q(k',b',z) - k'}_{\text{cash dividend if } (+) \text{ or equity issuance if } (-)} \\ 
&- \underbrace{T^d(w + b'q(k',b',z) - k')}_{\text{taxes on cash dividend}} \\
&- \underbrace{\Lambda(-(w + b'q(k',b',z) - k'))}_{\text{equity issuance cost}} \\
&+ \frac{1}{1+r(1-\tau_i)} E\Big[\underbrace{\max\{ V(w', z'), 0\}}_{\text{if }V \text{ is } (-) \text{, firm can default}}| z\Big] \Bigg\} \\
\text{where }
\underbrace{y'}_{\text{taxable corporate income}} &\equiv \underbrace{z' (k')^\alpha}_{\text{operating profits}} - \underbrace{\delta k'}_{\text{depreciation}} - \underbrace{(1-q(k',b',z)) b'}_{\text{interest on debt}} \\
\underbrace{T^C(x)}_{\text{corporate income tax bill}} &\equiv 
\begin{cases} 
\tau_c^+ x, & \text{if }x > 0 \\
\tau_c^- x, & \text{if }x \le 0
\end{cases}\\
\underbrace{w'}_{\text{realized net worth}} &\equiv \underbrace{y' - T^C(y')}_{\text{after-tax corporate income}}+ \underbrace{k'}_{\text{capital}}  - \underbrace{q(k',b',z)b'}_{\text{debt principal}} \\
\underbrace{T^d(x)}_{\text{taxes on cash dividend}} &= 
\begin{cases}
\frac{\bar \tau_d}{\phi}(\phi x + e^{-\phi x} - 1), & x > 0 \\
0,& x \le 0
\end{cases} \\
\underbrace{\Lambda(x)}_{\text{equity issuance cost}} &= 
\begin{cases} 
\lambda_0 + \lambda_1 x + \lambda_2 x^2, & \text{if }x > 0 \\
0, & \text{if }x \le 0
\end{cases}
\end{align*}

\pagebreak

\item Zero profit condition:

\begin{align*}
b' (1+r(1-\tau_i)) = (1+(1-\tau_i)\tilde r(k', b', z)) b' \int_{z_d(k', b', z)}^{\bar z} Q(z, dz') + \int_{\underline z}^{z_d(k', b', z)} R(k', z') Q(z, dz')
\end{align*}

\end{itemize}

\end{document}




