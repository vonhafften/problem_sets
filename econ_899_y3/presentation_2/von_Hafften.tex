\documentclass[usenames,dvipsnames, handout, aspectratio=169]{beamer}
\usetheme{Boadilla}
\usepackage{graphicx}
\usepackage{soul}
\usepackage{multirow}
\usepackage{multicol}
\usepackage{csvsimple}


\newcommand{\R}{\mathbb{R}}
\newcommand{\ubar}[1]{\underaccent{\bar}{#1}}
\newcommand{\Int}{\text{Int}}
\newcommand{\xbf}{\mathbf{x}}
\newcommand{\Abf}{\mathbf{A}}
\newcommand{\Bbf}{\mathbf{B}}
\newcommand{\Gbf}{\mathbf{G}}
\newcommand{\bbf}{\mathbf{b}}
\newcommand{\Lfn}{\mathcal{L}}
\newcommand{\one}{\mathbbm{1}}

\title[Firm Default with Long-Term Debt]{Firm Default with Long-Term Debt}
\author{Alex von Hafften}
\institute{UW-Madison}

\begin{document}

\begin{frame}
\titlepage
\end{frame}

\section{Introduction}

\begin{frame}[label = motivation]
\frametitle{Motivation}
\small
\begin{itemize}
\item Firm bankruptcy is important for firm dynamics through debt prices

\item Corbae and D'Erasmo (2021) extend structural corporate finance models {\color{gray}(i.e., Gomes (2001), and Hennessy and Whited (2007))} by adding Ch. 7 liquidation and Ch. 11 reorganization bankruptcy

\item All debt in CD (2021) is short-term in the sense that it matures before new debt can be issued

\item Long-term debt allows debt dilution; issuing new debt reduces value of existing debt obligations

\item But long-term debt increases computational complexity {\color{gray}(i.e., adds state variable)}

\item Arellano (2008) sovereign default model is related to Corbae and D'Erasmo (2021) \hyperlink{equilibirum_default_models}{\beamerbutton{More}}

\item Bornstein (2020) recasts Arellano (2008) in continuous time, uses methods of Achdou et al (2022), and then adds long-term debt a la Chatterjee and Eyigungor (2012)

\item I want to apply Bornstein (2020) approach to Corbae and D'Erasmo (2021)
\end{itemize}
\end{frame}


\begin{frame}[label = cts_time_methods]
\frametitle{Continuous Time Methods and Incomplete Markets}
\small
\begin{itemize}
\item Growing use of continuous time methods to incomplete-markets models
\begin{itemize}
{
\color{gray}
\item Financial frictions (Brunnermeier and Sannikov 2014)
\item Dynamics of inequality (Gabaix et al 2016)
}
\end{itemize}

\item Achdou et al (2022) develop efficient and portable algorithm to solve Aiyagari-Bewley-Huggett style models in continuous time

\item Recent applications of Achdou et al (2022) methods
\begin{itemize}
{
\color{gray}
\item Monetary policy with heterogeneous agents (Kaplan et al 2016)
\item Mortgage refinancing (Laibson et al 2020)
\item Durable consumption (McKay and Wieland 2021)
\item Precautionary savings with housing (Guerrieri et al 2020)
\item Marital wage premia (Pilossoph and Wee 2021)
}
\end{itemize}

\item Bornstein (2020) is only strategic equilibrium default model to use Achdou et al (2022) methods
\end{itemize}
\end{frame}



\begin{frame}[label = achdou_summary]
\frametitle{Achdou et al (2022) Summary \hyperlink{achdou_details_1}{\beamerbutton{Details}}}
\small
\begin{itemize}
\item Recast Aiyagari-Bewley-Huggett in continuous time 
\item Key advantages over discrete time: 
\begin{enumerate}
\item Optimal for assets to evolve continuously
\item Simultaneous change in asset and change in earning state is measure 0 event
\end{enumerate}
\item From (1), agent in interior knows she will reoptimize before hitting BC. So given guess of HJB, BC ``disappears" for agents in interior and can ``brute force" asset decision at BC
\item From (1) and (2), in an instant in the future,  agent at $(a, y)$ can be in at most four states
\begin{itemize}
\item One asset grid point higher and same earning state: $(a + \Delta a , y)$
\item Same level of assets and same earning state: $(a, y)$
\item One asset grid point lower and same earning state: $(a - \Delta a, y)$
\item Same level of assets and different earning state: $(a, \tilde y)$
\end{itemize}
So, transition matrix is sparse and near block diagonal $\implies$ easy to invert 
\item Anecdotal experience: 
\begin{itemize}
\item My discrete time Huggett (from ECON 899) took about 40 seconds
\item My version of Achdou et al (2022) took about 0.1 second
\item Achdou et al (2022) is ``less robust"
\end{itemize}
\end{itemize}
\end{frame}



\begin{frame}[label = bornstein_summary_1]
\frametitle{Bornstein (2022) with Short-Term Debt}
\small
\begin{itemize}
\item  As in Arellano (2008), a sovereign faces fluctuations in domestic output and can borrow or save a constant world interest rate to smooth domestic consumption
\item At any time, the sovereign can default on debt obligations and be excluded from world financial markets for a stochastic period of time and face output losses
\item Sovereign problem is close to Achdou et al (2022) but no BC and $r$ depends on state variables
\begin{align*}
E_0 \int_0^\infty & e^{-\rho t} u(c_t) dt \\
\text{s.t. } 
\dot a_t &= y_t  - c_t + r(y_t, a_t) a_t & \text{out of financial autarky}\\
0 &= y_t  - \phi(y_t) - c_t & \text{in financial autarky}
\end{align*}
where $y_t$ follows compound Poisson process with arrival rate and distribution $F(y', y)$ and $\phi(\cdot)$ is output loss in autarky
\item  Jumps are critical for endowment process for sovereigns (or productivity process for firms)
\begin{itemize}
\item  Endowment process is diffusion $\implies$ probability of default on default frontier is one $\implies$ interest rate on default frontier is infinity $\implies$ sovereign never defaults in equilibrium
\end{itemize}
\end{itemize}
\end{frame}

\begin{frame}[label = bornstein_summary_2]
\frametitle{Bornstein (2022) with Short-Term Debt (con't) }
\small
\begin{itemize}
\item  HJB of sovereign in autarky
\begin{align*}
\rho w(y) = u(y-\phi(y)) + \lambda_y\int_0^\infty (w(y')-w(y)) f(y', y) dy' + \lambda_D [v(0,y) - w(y)]
\end{align*}
where sovereign gets out of autarky with Poisson intensity $\lambda_D$
\item HJB of sovereign out of autarky
\begin{align*}
\rho v(a, y) = \max_c \Bigg\{ \rho w(y), u(c) + v_a(a, y) [y - c + r(a, y)a] + \lambda_y\int_0^\infty (v(a,y')-v(a,y)) f(y', y) dy'\Bigg\}
\end{align*}
\item Zero profit condition for outside investor
$$
r(a, y) - \lambda_y \int_0^\infty D(a, y') f(y',y) dy' = r^f
$$
where $D(a, y')$ indicates whether sovereign defaults 
\item Follow combination of Achdou et al (2022) and Hennessy and Whited (2007) algorithms to solve
\end{itemize}
\end{frame}


\begin{frame}[label = bornstein_summary_3]
\frametitle{Why is long-term debt pricing easier in continuous time?}
\small
\begin{itemize}
\item Bornstein (2020) sets up problem with long-term debt a la Chatterjee and Eyigungor (2012)
\item Long-term bonds mature with Poisson intensity $\lambda_b$ and pays flow payment $z$ before maturing
\item Flow budget constraint 
\begin{align*}
c_t  + q(a_t, y_t) \lambda_b a_t + q(a_t, y_t) \dot a_t &= y_t + za_t + \lambda_b a_t\\
\implies
s_t \equiv \dot a_t &= \frac{y_t + (z + \lambda_b) a_t - c_t}{q(y_t, a_t)} - \lambda_b a_t
\end{align*}
\item Zero profit condition
\begin{align*}
q(a, y) = \frac{1}{r^f}\Bigg[z + \lambda_b(1-q(a, y)) + \lambda_y \int_0^\infty [q(a, y') - q(a, y)] dF(y'|y) + s(a, y) q_a(a, y) \Bigg]
\end{align*}
\item In discrete time, we have to solve a fixed point problem
\item In continuous time, we can use finite differences and sparse matrix inversion
\end{itemize}
\end{frame}


\begin{frame}[label = conclusion]
\frametitle{Conclusion}
\small
\begin{itemize}
\item Today, I discussed recent methods for continuous time incomplete-markets models
\item I discussed how these methods are applied in equilibrium sovereign default models
\item Long-term debt is easier to work with in these models
\item My plan going forward is to try to implement this approach in a model looking at firm dynamics/bankruptcy/liquidation/reorganization problem
\end{itemize}
\end{frame}

\begin{frame}[label = equilibirum_default_models]
\frametitle{Equilibrium Strategic Default Models}
\small
\begin{center}
\begin{tabular}{ |c|c|c| } 
 \hline
 &&\\
 \textbf{Topic}    & \textbf{Discrete Time}     & \textbf{Continuous Time} \\ 
 &&\\
 \hline
 &&\\
 Consumer Default  & Chatterjee et al (2007)    & --- \\ 
 &&\\
 \hline
 &&\\
 Sovereign Default & Arellano (2008)            & Bornstein (2020) \\
  & Chatterjee and Eyigungor (2012)            &  \\
 &&\\ 
 \hline
 &&\\
 Firm Default      & Corbae and D'Erasmo (2021) & --- \\ 
 &&\\
 \hline
\end{tabular}
\end{center}
\hyperlink{motivation}{\beamerbutton{Back}}
\end{frame}






\begin{frame}[label = achdou_details_1]
\frametitle{Achdou et al (2022) Details}
\small
\begin{itemize}
\item HH maximize lifetime expected utility subject to exogenous BC
\begin{align*}
E_0 \int_0^\infty & e^{-\rho t} u(c_t) dt \\
\text{s.t. } \dot a_t &= y_t  - c_t + r a_t\\
a_t &\ge \underline a
\end{align*}
where $y_t$ follows Poisson process with two states $y_L<y_H$ with arrival rate $\lambda$
\item Stationary Hamilton-Jacob-Bellman equation
\begin{align*}
\rho V(a, y) =& \max_c \{u(c) + \dot a V_a(a, y) \} + \lambda [V(a, \tilde y) - V(a, y)]
\end{align*}
\item FOC: $u_c(c) =V_a(a, y) \implies c(a, y) = (u_c)^{-1}(V_a(a, y))$
\end{itemize}
\hyperlink{achdou_summary}{\beamerbutton{Back}}
\end{frame}


\begin{frame}[label = achdou_details_2]
\frametitle{Achdou et al (2022) (con't) Details}
\small
\begin{itemize}
\item Discretize $\implies V_{i,j}$ is HJB and $\partial V_{i,j}$ is partial derivative of HJB wrt $a$ at grid point $(a_i, y_j)$
\item Given $V_{i,j}$, approximate derivative using finite forward and backward differences
\begin{align*}
\partial V_{i,j}^F &\equiv \frac{V_{i,j} - V_{i+1,j}}{\Delta a},  \;\;\;
\partial V_{i,j}^B \equiv \frac{V_{i,j} - V_{i-1,j}}{\Delta a}
\end{align*}
\item FOCs implies asset drift for each finite difference
\begin{align*}
\dot a_{i,j}^F = y_i + ra_i - (u_c)^{-1}(\partial V_{i,j}^F),  \;\;\;
\dot a_{i,j}^B = y_i + ra_i - (u_c)^{-1}(\partial V_{i,j}^B)
\end{align*}
\item $V,u$ are concave $\implies \partial V_{i,j}^F < \partial V_{i,j}^B \implies \dot a_{i,j}^F < \dot a_{i,j}^B$
\item Use ``upwinding" rule to chose which finite difference:
\begin{align*}
\partial V_{i,j} \equiv 
\begin{cases} 
\partial V_{i,j}^F,  &\text{if } \dot a^B_{i,j} > \dot a^F_{i,j} > 0\\
\partial V_{i,j}^B,  &\text{if } 0 > \dot a^B_{i,j} > \dot a^F_{i,j}\\
\partial V_{i,j}^0,  &\text{if } \dot a^B_{i,j} \ge 0 \ge \dot a^F_{i,j}
\end{cases}
\end{align*}
where $\dot a_{i,j}^0 \equiv 0 \implies c_{i,j}^0 = y_j + ra_i \implies \partial V_{i,j}^0 \equiv u_c(y_j + r a_i)$
\item Brute force at BC $\implies \partial V_{1,j}^B = u_c(y_j + r a_1)$
\end{itemize}
\hyperlink{achdou_summary}{\beamerbutton{Back}}
\end{frame}

\begin{frame}[label = achdou_details_3]
\frametitle{Achdou et al (2022) (con't) Details}
\small
\begin{itemize}
\item How to update HJB guess? Intuition is to update nonstationary HJB until it converges:
\begin{align*}
\frac{\partial V(t,a,y)}{\partial t} + \rho V(t, a, y) =& \max_c \{u(c) + \dot a V_a(t, a, y) \} + \lambda [V(t, a, \tilde y) - V(t, a, y)]
\end{align*}
\item Natural first guess $\implies$ the ``explicit method"
\begin{align*}
\frac{V_{i,j}^{n+1} - V_{i,j}^n}{\Delta} + \rho V_{i,j}^n = u(c_{i,j}^n) + \dot a_{i,j}^n \partial V_{i,j}^n + \lambda [V_{i,-j}^n - V_{i,j}^n]
\end{align*}
However, explicit method is ``badly behaved"
\item ``Implicit method" works better
\begin{align*}
\frac{V_{i,j}^{n+1} - V_{i,j}^n}{\Delta} + \rho V_{i,j}^{n+1} = u(c_{i,j}^n) + \dot a_{i,j}^n \partial V_{i,j}^{n+1} + \lambda [V_{i,-j}^{n+1} - V_{i,j}^{n+1}]
\end{align*}
\end{itemize}
\hyperlink{achdou_summary}{\beamerbutton{Back}}
\end{frame}


\begin{frame}[label = achdou_details_4]
\frametitle{Achdou et al (2022) (con't) Details}
\small
\begin{itemize}
\item Implicit method results in closed form expression for updating HJB:
\begin{align*}
\frac{V^{n+1} - V^n }{\Delta} + \rho V^{n+1} &= u^n + \mathbf{A}^n V^{n+1}\\
\text{where } 
V^n &= [V_{1,1}^n, ..., V_{N,1}^n, V_{1,2}^n, ..., V_{N,2}^n]^T \\
u^n &= [u(c_{1,1}^n), ..., u(c_{N,1}^n), u(c_{1,2}^n), ..., u(c_{N,2}^n)]^T\\
\mathbf{A}^n &=
\begin{bmatrix}
y_{1,1} & z_{1,1} & 0       & ... & 0      & 0 & \lambda & 0       & ... & 0\\
x_{2,1} & y_{2,1} & z_{2,1} & ... & 0      & 0 & 0       & \lambda & ... & 0\\ 
\vdots  & \vdots  & \vdots  &     & \vdots & \vdots  &\vdots  & \vdots  &     & \vdots  \\
0       & 0       &  0       &  ...  & x_{N_a,1} & y_{N_a,1}  & 0  &  0   &...& \lambda \\
\lambda & 0       & 0 &... & 0 & 0 &y_{1,2} & z_{1,2} & ...       & 0      \\
0       & \lambda & 0 &... & 0 & 0 &x_{2,2} & y_{2,2} & ... & 0      \\ 
\vdots  & \vdots  & \vdots& \vdots  &    & \vdots & \vdots  &\vdots  &   &   \vdots  \\
0       &  0      & 0 &  ...& 0 & \lambda       & 0       &  0       &  ...  & y_{N_a,2} 
\end{bmatrix}
\end{align*}
\item $\mathbf{A}^n$ is large but it is sparse with at most 4 nonzero entries per row
\item Solving for the stationary distribution essentially boils down to inverting $A^n$
\end{itemize}
\hyperlink{achdou_summary}{\beamerbutton{Back}}
\end{frame}



\end{document}

