\documentclass{article}
\usepackage{amsmath,amsthm,amssymb,amsfonts}
\usepackage{setspace,enumitem}
\usepackage{graphicx}
\usepackage{hyperref}
\usepackage{natbib}
\usepackage{afterpage}
\usepackage{xcolor}
\usepackage{etoolbox}
\usepackage{booktabs}
\usepackage{pdfpages}
\usepackage{multicol}
\usepackage{geometry}
\usepackage{accents}
\usepackage{bbm}
\usepackage{placeins}
\hypersetup{
	colorlinks,
	linkcolor={blue!90!black},
	citecolor={red!90!black},
	urlcolor={blue!90!black}
}

\newtheorem{theorem}{Theorem}
\newtheorem{assumption}{Assumption}
\newtheorem{definition}{Definition}
\newtheorem{lemma}{Lemma}
\setlength{\parindent}{0cm}
\geometry{margin = 1in}

\newcommand{\R}{\mathbb{R}}
\newcommand{\ubar}[1]{\underaccent{\bar}{#1}}
\newcommand{\Int}{\text{Int}}
\newcommand{\xbf}{\mathbf{x}}
\newcommand{\Abf}{\mathbf{A}}
\newcommand{\Bbf}{\mathbf{B}}
\newcommand{\Gbf}{\mathbf{G}}
\newcommand{\bbf}{\mathbf{b}}
\newcommand{\one}{\mathbbm{1}}

\newtoggle{extended}
\settoggle{extended}{false}

\title{ECON 810: Homework 2}
\author{Alex von Hafften }

\begin{document}

\maketitle

\begin{itemize}

\section{Data}

\item I use the PSID with the following filters:

\begin{itemize}
\item Main sample, no SEO oversample.
\item Observations between 1978 and 1997, inclusive.
\item Ages 25 to 26, inclusive.
\end{itemize}

\item To define the job loss indicator:

\begin{itemize}
\item Dropping observations with lagged hours by one, two, and three periods under 1800 hours (=(52 weeks $-$ 2 weeks vacation) $\times$ 36 hours per week.
\item Dropping observations with lag size not equal to current size.
\item Job loss indicator is one if as hours in the current period being less than 75 percent of the average number of hours in the last three periods.
\end{itemize}

\item The resulting sample has 26,107 observations without job loss and 2,278 observations with job loss.

\item I then estimate a distributed lag regression around job loss.  The coefficients on job losses are plotted in the figure below.

\begin{itemize}
\item There's evidence of a pre-trend with the coefficients statistically significantly negative, around between negative \$2,000 to \$4,000.  I think this pre-trend is likely due to not limiting our analysis to ``mass layoffs" like Davis and von Wachter that can be identified using an employee-employer matched dataset.  We're capturing all job losses, so it reasonable to suspect that losing a job might not be a fully exogenous event.
\item The drop in earning from the period before job loss $k = -1$ to the period after job loss $k = 1$ is about \$9,700.  This estimate is smaller than Davis and von Wachter's estimate for both an expansion and a recession. I think this estimate being smaller could also be due to the definition of job loss being ``less exogenous" here than in Davis and von Wachter.  Indeed, the estimate after job loss $k=1$ relative to the control group is about \$14,100, which is more in align is Davis and von Wachter's estimates.

\end{itemize}

\item See \texttt{part\_1.R} for implementation.

\end{itemize}

\includegraphics{part_1.png}

\pagebreak

\section{Model}

Since the matching function $M$ is CRS, we can rewrite the job finding rate and the hiring rate in terms of market tightness $\theta \equiv \frac{v}{u}$ where $v$ is vacancies and $u$ is applicants. The job-finding rate becomes:

\begin{align*}
p(\theta) = \frac{M(u, v)}{u} = M(1, \frac{v}{u}) = M(1, \theta) = \frac{\theta}{(1+\theta^\zeta)^{1/\zeta}}
\end{align*}

The hiring rate becomes:

\begin{align*}
p_f(\theta) = \frac{M(u, v)}{v} =M( \frac{u}{v}, 1) = M(\theta^{-1}, 1) = \frac{\theta^{-1}}{(\theta^{-\zeta}+ 1)^{1/\zeta}} = \frac{1}{(1+\theta^\zeta)^{1/\zeta}}
\end{align*}

The inverse of the hiring rate is:

\begin{align*}
\theta(p_f) = (p_f^{-\zeta} - 1)^{1/\zeta}
\end{align*}

\begin{enumerate}

\item Write down the definition of equilibrium for this economy.

An recursive competitive equilibrium is

\begin{itemize}

\item Market tightnesses $\{\theta_t(\omega, h): t \in \{1, ..., 120\}, \omega \in [0, 1], h \in [0.5, 1.5]\}$,

\item Job search decisions $\{\omega_t(b, h): t \in \{1, ..., 120\}, h \in [0.5, 1.5], b \in \R_+\}$,

\item Savings decisions for unemployed workers $\{b^U_t(b, h): t \in \{1, ..., 120\}, h \in [0.5, 1.5], b \in \R_+\}$,

\item Savings decision for employed workers $\{b_t^W(\omega, b, h): t \in \{1, ..., 120\}, \omega \in [0, 1], h \in [0.5, 1.5], b \in \R_+\}$, 

\item A distribution of unemployed workers $\{\lambda_t^U(b, h): t \in \{1, ..., 120\}, h \in [0.5, 1.5], b \in \R_+\}$, 

\item A distribution of employed workers $\{\lambda_t^W(\omega, b, h): t \in \{1, ..., 120\}, \omega \in [0, 1], h \in [0.5, 1.5], b \in \R_+\}$ 
 
\end{itemize}

such that

\begin{itemize}

\item The free-entry of firms pin down market tightnesses.

\item Given market tightnesses, the job search decisions are optimal.

\item Savings decisions are optimal.

\item The distribution of unemployed and employed workers is consistent with policy functions.

\end{itemize}

\pagebreak
 
\item Prove that the equilibrium is \textit{Block Recursive}.

We start in the terminal period. In the terminal period, the firm value is not a function of the distribution of workers: $J_T(h, \omega) = (1-\omega) f(h)$. By the free entry condition, 

$$
\theta_T(h, \omega) J_T(h, \omega) \le \kappa
$$

So market tightness in the terminal period is not a function of the distribution of workers.  All workers eat everything in the terminal period, so their value functions are also not functions of the distribution of workers:

\begin{align*}
W_T(\omega, b, h) &= u((1-\tau)\omega f(h) + b) \\
U_T(\omega, b, h) &= z + b \\
\end{align*}

Now we proceed by induction. Assume that decisions in period $t+1$ do not depend on the distribution of workers.  Since the firm value function does not depend on the distribution of workers, the market tightness in $t$ does not depend on the distribution of workers.. Since the unemployed worker Bellman does not depend on the distribution of workers, the optimal decisions of the unemployed worker does not depend on the distribution of workers.  This similarly holds for the employed worker.  Thus, the equilibrium is block recursive.

\item Solve the model above using the suggested parameters and report the following:

\bigskip

One note on my approach: I was running into uninteresting simulation results when I started all simulations from the minimum human capital level, so for each simulation I drew a random starting point between 0.5 and 1.5.

\bigskip

\begin{enumerate}

\item Make a histogram of the distribution of assets from your simulated data.

\begin{center}
\includegraphics[scale = 0.5]{figure_3a}
\end{center}

\pagebreak

\item Make a histogram of the distribution of wages from your simulated data.

\begin{center}
\includegraphics[scale = 0.5]{figure_3b}
\end{center}

\item What is the unemployment rate in your economy?

\bigskip

The unemployment rate is 48 percent.

\bigskip

\item Plot average earnings and assets over the life-cycle from your model. How do the earnings data compare to your age earnings profile from the PSID?

\begin{center}
\includegraphics[scale = 0.5]{figure_3d_1}
\includegraphics[scale = 0.5]{figure_3d_2}
\end{center}

The age earnings profile from the PSD is much different compared the earnings profile here.  Average earnings steadily fall here whereas in the PSID the earnings follow a hump shape that peaks in middle age.

\bigskip

\item In your simulated data, what is the average gain in earnings when employed? How does this compare to your data estimate from last week's assignment?

\bigskip

The average gain while employed is about 0.5 percent.  That is much smaller than my estimate from last week's assignment.  Although I did not adjust my data estimates for inflation.

\bigskip

\item In your simulated data, make a graph of earnings around job loss (following an individual for 4 quarters before job loss to 8 quarters after job loss). How does the graph compare to the estimates from Davis and von Wachter and your estimates from part of the assignment?

\bigskip

This graph more-or-less resembles the Davis and von Wachter graph, but the the initial recovery is much faster.  The earnings drop to the unemployment benefit level and quickly recover at first, but earnings stay about 15 percent lower for an extended period.

\begin{center}
\includegraphics[scale = 0.5]{figure_3f}
\end{center}

\pagebreak

\item In your simulated data, make a graph of consumption around job loss.

\begin{center}
\includegraphics[scale = 0.5]{figure_3g}
\end{center}

\item Increase the transfer to the unemployed by 10\%. How do the above graphs of earnings and consumption around job loss change? What happens to the unemployment rate in your economy?

\begin{center}
\includegraphics[scale = 0.5]{figure_3h_1}
\includegraphics[scale = 0.5]{figure_3h_2}
\end{center}

The unemployment rate is higher at about 50 percent.

\end{enumerate}

\end{enumerate}

\end{document}

