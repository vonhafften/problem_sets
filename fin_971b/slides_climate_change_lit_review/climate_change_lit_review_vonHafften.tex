\documentclass{beamer}
\usetheme{Boadilla}
\usepackage{graphicx}
\usepackage{bbm}
\newcommand{\one}{\mathbbm{1}}
\title[Climate Change and Corporate Finance]{Recent Developments in \\ Climate Change and Corporate Finance}

\author{Alex von Hafften}
\institute{UW-Madison}

\begin{document}

\begin{frame}
\titlepage
\end{frame}

\begin{frame}
\frametitle{Motivation}
\begin{itemize}[<+->]
\item The economic and financial implications of climate change is a growing area of research, including within corporate finance.
\bigskip
\item The most recent programs of all top five financial conferences had presentations on related topics.
\bigskip
\item Four out of five had entire sessions dedicated to it.
\end{itemize}
\end{frame}



\begin{frame}
\frametitle{Methodology}
\begin{itemize}[<+->]
\item In this literature review, I summarized the recent development in the intersection of climate change and corporate finance.
\bigskip
\item The starting point is related papers in the most recent programs of the following finance conferences:
\begin{enumerate}
\item Western Finance Association Meetings 2021
\item American Finance Association Meetings 2022
\item Society for Financial Studies Cavalcade North America 2021
\item Utah Winter Finance Conference 2022
\item NBER Corporate Finance Summer Institute 2021
\end{enumerate}
\bigskip
\item All told, there ended up being twenty-seven related papers from these conferences alone.
\end{itemize}
\end{frame}


\begin{frame}
\frametitle{Terminology}
\begin{itemize}
\item ESG = ``Environmental, Social, and Governance"
\item E\&S = ``Environmental and Social"
\item SRI = ``Socially Responsible Investing"
\item CSR = ``Corporate Social Responsibility"
\item Impact investing

\bigskip

\item ``Green investors" or ``social investors" are investors who care about monetary and non-monetary payoffs.
\item ``Traditional investors" or ``commercial investors" are investors who only care about monetary payoffs.

\bigskip

\item CEI = ``Carbon Emission Intensity"
\item GHG = ``Greenhouse Gas"
\end{itemize}
\end{frame}


\begin{frame}
\frametitle{Motivating Facts}
\begin{itemize}[<+->]
\item As of 2020, one-third of U.S. assets under management (\$17 trillion) have some sort of ESG objective.
\bigskip
\item Possible change in the ``social contract" for firms.
\begin{itemize}
\item Friedman (1970): ``The social responsibility of business is to increase its profits."
\item Business Roundtable (2019): The purpose of a corporation is to promote ``an economy that serves all Americans."
\end{itemize}
\end{itemize}
\end{frame}



\begin{frame}
\frametitle{Major Questions}
\begin{itemize}[<+->]
\item How to measure a firm's social impact?
\bigskip
\item How should we characterize investors' preferences about firm ethical behavior?
\bigskip
\item How do investors with heterogenous preferences affect asset prices and capital allocation?
\bigskip
\item How costly is ESG investing relative to traditional investing?
\bigskip
\item What does ESG investing look like outside of public equity?
\bigskip
\item What are the real effects of ESG investing?
\end{itemize}
\end{frame}



\begin{frame}
\frametitle{How to measure a firm's ESG impacts?}
There's a lot of ESG rating agencies with ad hoc methodology (weighted average of emissions, board diversity, etc.) and conflicting results.
\bigskip
\begin{itemize}[<+->]
\item Using survey data, Allcott et al (2021) measure social impact as the social welfare loss from a firm's exit in equilibrium. Consumer surplus dominates profits, worker surplus, and externalities. Their ratings are largely orthogonal to existing ESG ratings.
\bigskip
\item Huang et al (2020) measure ESG impact based on firm's internet search intensity around ESG-related topics. Increases in attention to ESG topics predicts
improvements in that firm's ESG ratings.
\bigskip
\item Sautner et al (2021) measure a firm's climate change exposures using text analysis of earnings conference calls.
\bigskip
\item Berg et al (2021) refine ESG ratings using classical errors-in-variables approach with ESG ratings from other agencies as instruments. OLS of stock prices on ESG ratings are biased downward by 60\% compared to 2SLS (attenuation bias). Average signal-to-noise is 60\%.
\end{itemize}
\end{frame}



\begin{frame}
\frametitle{How to measure a firm's ESG impacts? (con't)} 
What are the effects of more detailed mandatory ESG disclosures?
\bigskip
\begin{itemize}[<+->]
\item Kreuger et al (2021) find that mandatory ESG reporting improves firm's information
environment (analysts' earnings forecasts become more accurate and less dispersed), negative ESG incidents become less likely, and stock price crash risk declines.
\bigskip
\item Goldstein et al (2021) show that an improvement in
the quality of non-monetary information can reduce overall price informativeness for traditional investors
and increase firm's cost of capital. 
\end{itemize}
\end{frame}



\begin{frame}
\frametitle{How should we characterize investors' preferences about firm ethical behavior?}
A couple paper use experiments to learn about how investors' ``moral" preferences about firm behavior.
\bigskip
\begin{itemize}[<+->]
\item Bonnefon et al (2021) find participants are willing to pay \$0.70 more for buying a share in a firm that gives one more dollar per share to charities. Symmetrically, a firm that makes profits by exercising a negative externality of \$1 on a charity is valued \$0.90 less. Scaling of non-pecuniary preferences is linear.
\bigskip
\item Heeb et al (2021) find that investors have a higher WTP for a sustainable investment, but it does not grow with the social impact of the investment.
\bigskip
\item Colonnelli and Gormsen (2021) find evidence of ``big business discontent" using perceptions of ESG impact.  They find that higher discontent leads to lower support for corporate bailouts.
\end{itemize}
\end{frame}


\begin{frame}
\frametitle{How do investors with heterogenous preferences affect asset prices?}
\bigskip
\begin{itemize}[<+->]
\item Goldstein et al (2021) create a rational expectation equilibrium model with two types of investors: green and traditional. Heterogeneous preferences contaminate price informativeness to different type.  
\begin{itemize}
\item Positive signal about non-monetary payoff $\rightarrow$ increase green investor demand $\rightarrow$ traditional investors cut back demand because they infer from  the price a worse realization of the monetary payoff.
\end{itemize}
\end{itemize}
\end{frame}




\begin{frame}
\frametitle{How do investors with heterogenous preferences affect capital allocation?}
\bigskip
\begin{itemize}[<+->]
\item Green and Roth (2020) argue against ``value-aligned" investment strategies for green investors:
\begin{itemize}
\item Firm A generates a 10\% profit and 10 units of social value.
\item Firm B generates a 8\% profit and 5 units of social value.
\item Firm C generates a 9\% profit and 0 units of social value.
\end{itemize}
``Value-aligned" strategy would be to invest in Firm A, then commercial investor invests in Firm C. Better to invest in Firm B and allow commercial investor to invest in Firm A.
\bigskip
\item Landier (2021) argue that a socially responsible fund should prioritize investments in
companies with acute negative externalities and facing strong capital search friction that commit to capping their emissions.
\end{itemize}
\end{frame}



\begin{frame}
\frametitle{How costly is ESG investing?}
\bigskip
\begin{itemize}[<+->]
\item Lindsey et al (2021) find that implementing ESG strategies in equities sacrifices negligible profits.
\bigskip
\item Lo and Zhang (2021) derive conditions under which impact investing detracts from, improves on, or is neutral to the performance of traditional mean-variance optimal portfolios.  These conditions depend on whether the correlation between the impact and unobserved excess return are negative, positive, or zero, respectively.
\bigskip
\item Pastor et al (2021) construct a ``green factor", a return spread between environmentally friendly and unfriendly stocks. They show that U.S. green stocks outperformed peers as climate concerns strengthened, but their positive performance would disappear without climate-concern shocks.
\end{itemize}
\end{frame}



\begin{frame}
\frametitle{What does ESG investing look like outside of public equity?}
ESG investing has been until recently been confined to public equity markets but has moved into corporate bonds, options, and bank lending.
\bigskip
\begin{itemize}[<+->]
\item Diep et al (2021) find that there are only modest distortions to incorporate ESG objectives in corporate bond portfolio, but ESG measures do not predict future credit excess returns. 
\bigskip
\item  Duan et al (2021) find the bonds of high CEI firms are riskier on average than those of low CEI firms (higher bond market beta, higher downside risk, higher illiquidity, and lower credit ratings). However, no evidence of a ``carbon risk premium" in pricing.
\bigskip
\item Cao et al (2021) find that uncertainty around ESG issues is priced in the option market.  The implied volatility is higher (thus, the option prices higher) for firms with poor ESG ratings.
\bigskip
\item Ivanov et al (2021) estimate how CEI affect bank loans.  High-emission firms face shorter
loan maturities, lower access to permanent forms of bank financing, higher interest
rates, and higher participation of shadow banks in their lending syndicates.
\end{itemize}
\end{frame}


\begin{frame}
\frametitle{What are the real effects of ESG investing?}
\begin{itemize}[<+->]
\item Gantchev et al (2021) find that negative news coverage of ESG risks $\rightarrow$ green investors divest $\rightarrow$ temporary decline in valuation $\rightarrow$ firms improve ESG policies.
\bigskip
\item Heath et al (2021) find that SRI funds select firms with higher E\&S standards, but there is no evidence that they improve firm behavior.
\bigskip
\item Naaraayan et al (2020) find that shareholder environmental activism leads to firms reducing pollution by taking on costly abatement initiatives.
\bigskip
\item Krueger (2021) find that firm's sustainability policies reduce labor costs and enable firms to recruit and retain high skilled workers. Workers earn about 10\%
lower wages in firms that operate in more sustainable sectors.
\end{itemize}
\end{frame}


\begin{frame}
\frametitle{What are the real effects of ESG investing? (con't)}
\begin{itemize}
\item Hong et al (2021) build a DSGE model to evaluate the welfare consequences of mandates to invest in sustainable firms. They argue that existing mandates are insufficient to achieve first best.
\end{itemize}
\end{frame}



\begin{frame}
\frametitle{Other Papers}
\begin{itemize}[<+->]
\item \textbf{Greenwashing:} Gibson et al (2020) find that U.S. institutional investors that make public commitments to responsible investing have portfolios with weakly worse ESG ratings.
\bigskip
\item \textbf{Effect of competition on pollution:} Grinstein and Larkin (2021) find that electric utilities polluted less following increased competition pressures. Utilities moved to cheaper and less polluting production processes and competition improved allocation across plants.
\bigskip
\item \textbf{Political ideology:} Kaviani et al (2021) find that the CSR rating of firms declined significantly after increased exposure to conservative media. Change in local ideology drives the results.
\end{itemize}
\end{frame}



\begin{frame}
\frametitle{References}
\scriptsize
\begin{itemize}
\item Allcott, Montanari, Tan. (2021) ``An Economic View of Corporate Social Impact." NBER Corporate Finance SI 2021 Working Paper.
\item Berg, Koelbel, Pavlova, Rigobon. (2021) ``ESG Confusion and Stock Returns: Tackling the Problem of Noise." AFA Annual Meeting 2022 Working Paper.
\item Bonnefon, Landier, Sastry, Thesmar. (2021) ``Do Investors Care about Corporate Externalities? Experimental Evidence." AFA Annual Meeting 2022 Working Paper.
\item Cao, Goyal, Zhan, Zhang. (2021) ``Unlocking ESG Premium from Options." AFA Annual Meeting 2022 Working Paper.
\item Colonnelli, Gormsen. (2021) ``Selfish Corporations." WFA Meeting 2021 Working Paper.
\item Diep, Pomorski, Richardson. (2021) ``Sustainable Systematic Credit." AFA Annual Meeting 2022 Working Paper.
\item Duan, Li, Wen. (2021) ``Is Carbon Risk Priced in the Cross-Section of Corporate Bond Returns." SFS Cavalcade North America 2021 Working Paper.
"\item Gantchev, Giannetti, Li. (2021) ``Does Money Talk? Market Discpline through Selloffs and Boycotts." SFS Cavalcade North America 2021, AFA Annual Meeting 2022 Working Paper."
\item Brandon Gibson, Glossner, Krueger, Matos, Steffen. (2020) ``Do Responsible Investors Invest Responsibly?." SFS Cavalcade North America 2021 Working Paper.
\item Goldstein, Kopytov, Shen, Xiang. (2021) ``On ESG Investing: Heterogeneous Preferences, Information, and Asset Prices." SFS Cavalcade North America 2021, AFA Annual Meeting 2022 Working Paper.
\end{itemize}
\end{frame}


\begin{frame}
\frametitle{References (con't)}
\scriptsize
\begin{itemize}
\item Green, Roth. (2020) ``The Allocation of Socially Responsible Capital." SFS Cavalcade North America 2021, Utah Winter Conference 2022, WFA Meeting 2021 Working Paper.
\item Grinstein, Larkin. (2021) ``Minimizing Costs, Maximizing Sustainability." SFS Cavalcade North America 2021 Working Paper.
\item Heath, Macciocchi, Michaely, Ringgenberg. (2021) ``Does Socially Responsible Investing Change Firm Behavior?." AFA Annual Meeting 2022 Working Paper.
\item Heeb, Koelbel, Paetzold, Zeisberger. (2021) ``Do Investors Care about Impact?." AFA Annual Meeting 2022 Working Paper.
\item Hong, Wang, Yang. (2021) ``Welfare Consequences of Sustainable Finance." AFA Annual Meeting 2022 Working Paper.
\item Huang, Karolyi, Kwan. (2020) ``Paying Attention to ESG: Evidence from Big Data Analytics." SFS Cavalcade North America 2021, AFA Annual Meeting 2022 Working Paper.
\item Ivanov, Kruttli, Watugala. (2021) ``Banking on Carbon: Corporate Lending and Cap-and-Trade Policy." SFS Cavalcade North America 2021 Working Paper.
\item Kaviani, Li, Maleki. (2021) ``Media, Partisan Ideology, and CSR." AFA Annual Meeting 2022 Working Paper.
\item Krueger, Metzger, Wu. (2021) ``The Sustainability Wage Gap." AFA Annual Meeting 2022 Working Paper.
\item Krueger, Sautner, Tang, Zhong. (2021) ``The Effects of Mandatory ESG Disclosure around the World." AFA Annual Meeting 2022 Working Paper.
\item Landier, Lovo. (2021) ``Socially Responsible Finance: How to Optimize Impact?." AFA Annual Meeting 2022 Working Paper.
\end{itemize}
\end{frame}


\begin{frame}
\frametitle{References (con't)}
\scriptsize
\begin{itemize}
\item Lindsey, Pruitt, Schiller. (2021) ``The Cost of ESG Investing." AFA Annual Meeting 2022 Working Paper.
\item Lo, Zhang. (2021) ``Quantifying the Impact of Impact Investing." AFA Annual Meeting 2022 Working Paper.
\item Pastor, Stambaugh, Taylor. (2021) ``Dissecting Green Returns." AFA Annual Meeting 2022 Working Paper.
\item Sautner, van Lent, Vilkov, Zhang. (2021) ``Firm-level Climate Change Exposure." AFA Annual Meeting 2022 Working Paper.
\end{itemize}
\end{frame}

\end{document}

