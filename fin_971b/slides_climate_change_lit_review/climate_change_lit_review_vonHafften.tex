\documentclass[handout]{beamer}
\usetheme{Boadilla}
\usepackage{graphicx}
\usepackage{bbm}
\newcommand{\one}{\mathbbm{1}}
\title[Climate Change and Corporate Finance]{Recent Developments in \\ Climate Change and Corporate Finance}

\author{Alex von Hafften}
\institute{UW-Madison}

\begin{document}

\begin{frame}
\titlepage
\end{frame}

\begin{frame}
\frametitle{Motivation}
\begin{itemize}[<+->]
\item The economic and financial implications of climate change is a growing area of research, including within corporate finance.
\bigskip
\item The most recent programs of all top five financial conferences had presentations on related topics.
\bigskip
\item Four out of five had entire sessions dedicated to it.
\end{itemize}
\end{frame}



\begin{frame}
\frametitle{Methodology}
\begin{itemize}[<+->]
\item In this literature review, I summarized the recent development in the intersection of climate change and corporate finance.
\bigskip
\item The starting point is related papers in the most recent programs of the following finance conferences:
\begin{enumerate}
\item Western Finance Association Meetings 2021
\item American Finance Association Meetings 2022
\item Society for Financial Studies Cavalcade North America 2021
\item Utah Winter Finance Conference 2022
\item NBER Corporate Finance Summer Institute 2021
\end{enumerate}
\bigskip
\item All told, there ended up being twenty-eight related papers from these conferences alone.
\end{itemize}
\end{frame}


\begin{frame}
\frametitle{Terminology}
\begin{itemize}[<+->]
\item ESG = ``Environmental, Social, and Governance"
\item E\&S = ``Environmental and Social"
\item SRI = ``Socially Responsible Investing"
\item Impact investing

\bigskip
\item ``Green investors" or ``social investors" are investors who care about monetary and non-monetary payoffs.
\item ``Traditional investors" or ``commercial investors" are investors who only care about monetary payoffs.

\bigskip
\item CEI = ``Carbon Emission Intensity"
\end{itemize}
\end{frame}


\begin{frame}
\frametitle{Motivating Facts}
\begin{itemize}[<+->]
\item As of 2020, one-third of U.S. assets under management (\$17 trillion) have some sort of ESG objective.
\bigskip
\item Possible change in the ``social contract" for firms.
\begin{itemize}
\item Friedman (1970): ``The social responsibility of business is to increase its profits."
\item Business Roundtable (2019): The purpose of a corporation is to promote ``an economy that serves all Americans."
\end{itemize}
\end{itemize}
\end{frame}



\begin{frame}
\frametitle{Major Questions}
\begin{itemize}[<+->]
\item How to measure a firm's social impact?
\bigskip
\item How should we characterize investors' preferences about firm ethical behavior?
\bigskip
\item How do investors with heterogenous preferences affect asset prices and capital allocation?
\bigskip
\item What does ESG investing look like outside of public equity?
\bigskip
\item What are the real effects of ESG investing?
\end{itemize}
\end{frame}



\begin{frame}
\frametitle{How to measure a firm's ESG impacts?}
There's a lot of ESG rating agencies with ad hoc methodology (weighted average of emissions, board diversity, etc.) and conflicting results.
\bigskip
\begin{itemize}[<+->]
\item Using survey data, Allcott et al. (2021) measure social impact as the social welfare loss from a firm's exit in equilibrium. Consumer surplus dominates profits, worker surplus, and externalities. Their ratings are largely orthogonal to existing ESG ratings.
\bigskip
\item Berg et al. (2021) refine ESG ratings by using ESG ratings from other agencies as instruments (i.e. classical errors-in-variables). OLS of stock prices on ESG ratings are biased downward by 60\% compared to 2SLS (attenuation bias). Average signal-to-noise is 60\%.
\end{itemize}
\end{frame}



\begin{frame}
\frametitle{How should we characterize investors' preferences about firm ethical behavior?}
A couple paper use experiments to learn about how investors' ``moral" preferences about firm behavior.
\bigskip
\begin{itemize}[<+->]
\item Bonnefon et al. (2021) find participants are willing to pay \$0.70 more for buying a share in a firm that gives one more dollar per share to charities. Symmetrically, a firm that makes profits by exercising a negative externality of \$1 on a charity is valued \$0.90 less. Scaling of non-pecuniary preferences is linear.
\bigskip
\item Heeb et al. (2021) find that investors have a higher WTP for a sustainable investment, but it does not grow with the social impact of the investment.
\bigskip
\item Colonnelli and Gormsen (2021) find evidence of ``big business discontent" using perceptions of ESG impact.  They find that higher discontent leads to lower support for corporate bailouts.
\end{itemize}
\end{frame}


\begin{frame}
\frametitle{How do investors with heterogenous preferences affect asset prices and capital allocation?}
\bigskip
\begin{itemize}[<+->]
\item Goldstein et al. (2021) create a rational expectation equilibrium model with two types of investors: green and traditional. Heterogeneous preferences contaminate price informativeness to different type.  
\begin{itemize}
\item Positive signal about non-monetary payoff $\rightarrow$ increase green investor demand $\rightarrow$ traditional investors cut back demand because they infer from  the price a worse realization of the monetary payoff.
\end{itemize}
\bigskip
\item Green and Roth (2020) argue against ``value-aligned" investment strategies for green investors:
\begin{itemize}
\item Firm A generates a 10\% profit and 10 units of social value.
\item Firm B generates a 8\% profit and 5 units of social value.
\item Firm C generates a 9\% profit and 0 units of social value.
\end{itemize}
``Value-aligned" strategy would be to invest in Firm A, then commercial investor invests in Firm C. Better to invest in Firm B and allow commercial investor to invest in Firm A.
\end{itemize}
\end{frame}


\begin{frame}
\frametitle{What does ESG investing look like outside of public equity?}
ESG investing has been until recently been confined to public equity markets but has moved into corporate bonds and options.
\bigskip
\begin{itemize}[<+->]
\item Diep et al. (2021) find that there are only modest distortions to incorporate ESG objectives in corporate bond portfolio, but ESG measures do not predict future credit excess returns. 
\bigskip
\item  Duan et al. (2021) find the bonds of high CEI firms are riskier on average than those of low CEI firms (higher bond market beta, higher downside risk, higher illiquidity, and lower credit ratings). However, no evidence of a ``carbon risk premium" in pricing.
\bigskip
\item Cao et al. (2021) find that uncertainty around ESG issues is priced in the option market.  The implied volatility is higher (thus, the option prices higher) for firms with poor ESG ratings.
\end{itemize}
\end{frame}



\begin{frame}
\frametitle{What are the real effects of ESG investing?}
\begin{itemize}[<+->]
\item Gantchev et al. (2021) find that negative news coverage of ESG risks $\rightarrow$ green investors divest $\rightarrow$ temporary decline in valuation $\rightarrow$ firms improve ESG policies.
\bigskip
\item Heath et al. (2021) find that SRI funds select firms with higher E\&S standards, but there is no evidence that they improve firm behavior.
\bigskip
\item Naaraayan et al. (2020) find that shareholder environmental activism leads to firms reducing pollution by taking on costly abatement initiatives.
\end{itemize}
\end{frame}



\begin{frame}
\frametitle{Other Papers}
\begin{itemize}[<+->]
\item \textbf{Greenwashing:} Gibson et al. (2020) find that U.S. institutional investors that make public commitments to responsible investing have portfolios with weakly worse ESG ratings.
\bigskip
\item \textbf{Effect of competition on pollution:} Grinstein and Larkin (2021) find that electric utilities polluted less following increased competition pressures. Utilities moved to cheaper and less polluting production processes and competition improved allocation across plants.
\end{itemize}
\end{frame}

\end{document}

