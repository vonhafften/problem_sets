\documentclass{article}
\usepackage{amsmath,amsthm,amssymb,amsfonts}
\usepackage{setspace,enumitem}
\usepackage{graphicx}
\usepackage{hyperref}
\usepackage{natbib}
\usepackage{afterpage}
\usepackage{xcolor}
\usepackage{etoolbox}
\usepackage{booktabs}
\usepackage{pdfpages}
\usepackage{multicol}
\usepackage{soul}
\usepackage{geometry}
\usepackage{accents}
\usepackage{accents}
\hypersetup{
	colorlinks,
	linkcolor={blue!90!black},
	citecolor={red!90!black},
	urlcolor={blue!90!black}
}
\usepackage{setspace}

\newtheorem{theorem}{Theorem}
\newtheorem{assumption}{Assumption}
\newtheorem{definition}{Definition}
\newtheorem{proposition}{Proposition}
\newtheorem{lemma}{Lemma}
\newcommand{\R}{\mathbb{R}}
\newcommand{\N}{\mathbb{N}}
\newcommand{\Lfn}{\mathcal{L}}
\newcommand{\Int}{\text{Int}}
\newcommand{\ubar}[1]{\underaccent{\bar}{#1}}
\newcommand{\xbf}{\mathbf{x}}
\newcommand{\Abf}{\mathbf{A}}
\newcommand{\Bbf}{\mathbf{B}}
\newcommand{\Gbf}{\mathbf{G}}
\newcommand{\bbf}{\mathbf{b}}

\title{Referee Report on Chari and Kehoe (2016)}
\author{Alex von Hafften}
\date{\today}

\doublespacing

\begin{document}

\maketitle

\section{Summary}

This paper formalizes the idea that the lack of a government's commitment to not bailing out firms in a downturn creates inefficiencies in an otherwise efficient private market; these inefficiencies open the door to regulations that improve the outcome.  Chari and Kehoe setup a private market between the managers of firms and investors.  They consider a rich set of contract between these agents in face of information frictions: an effort moral hazard choice of the managers and private idiosyncratic productivity shocks.  The investor can put the firm through costly bankruptcy to observe the private idiosyncratic productivity akin to costly state verification from Townsend (1979). The contracts can be renegotiated after the realization of the aggregate productivity shock. The authors show that the optimal contracts boil down to equity-debt contracts.  The contracts are ex ante efficient and, with the renegotiation, they are also ex post efficient.  So, importantly, this private market is efficient.

\bigskip

The authors then introduce a government that can bailout distressed firms using a tax levied on all firms.  Ex ante, the government does not want to bailout distressed firms, so with commitment the government does not bail out firms. But if the government lacks commitment and bankruptcy is sufficiently costly, it will bailout all firms. Since managers do not perceive that their choices affect the taxes they must pay, they perceive the bailout to be akin to a subsidy to distressed firms. Thus, firm chooses to be larger (size distortion) and to put in lower effort (effort distortion).

\bigskip

Given the lack of commitment, regulations like limits on the debt-to-value ratio of firms and taxes on firm size can implement the constrained efficient outcome---``constrained" efficient in the sense that it respects the sustainability constraint from the lack of commitment by the government as well as resource and information constraints. Furthermore, orderly resolution authority, which is a major part of the post financial crisis regulation and allows the government to impose losses on unsecured creditor holds, can improve the sustainability inefficient outcome, but, by itself, does not implement the constrained efficient outcome.

\section{Relationship to in-class material and what I learned}

The contracting problem (section I.A) is very new relative to the material presented in class. It is similar to many of the contracting problem, we discussed in Dean's corporate finance class. However, the treatment of the infinite horizon dynamic model and the reversion (trigger) strategy is closer to what we discussed in class.  Specifically, the full bailout equilibrium is the ``worst case" static sustainable equilibrium.  Similar to the autarky equilibrium in the capital taxation model, neither the agents (managers and investors, here) nor the government have an incentive to deviate from this worst case.  Based on the reversion to the worst case static sustainable equilibrium, we consider the set of sustainable equilibria that can be support by reverting to that worst case in the case of deviation on part of the government.  The name of the game is weighing the benefit of the best one-shot deviation to the cost of reverting to the worst case for the rest of time.  In this sense, this paper is a clear application of the broad idea of Chari and Kehoe (1990) originally applied to a simple environment (i.e., the Fischer (1980) capital taxation model ) to a much more complicated environment (i.e., contracting problem in section I.A).

\bigskip

I worked at the Federal Reserve Board for five years prior to starting the Ph.D. program at Wisconsin.  This notion that the financial system would be efficient, but that government lacks the commitment technology to not bailout large financial institutions like banks, insurance companies, government sponsored enterprise during financial crisis was relatively well accepted, especially by the economists I worked for.  Instead of being articulated as a lack of government commitment, many economists thought of it as an implicit government guarantee. Some projects looked at trying to quantify the extent of these implicit government guarantees.  For me, this paper is nice formalization of this idea. Much of this work I was exposed to extended beyond banks---AIG, an insurance company, was bailed out in 2008 and Fannie Mae and Freddie Mac were placed into conservatorship by the Treasury.

\bigskip

$<<<$More about what I learned$>>>$

\section{Ideas that are influenced by this paper}

\textit{Quantifying}---My immediate thought is bringing this model to the data and thinking about how to answer related quantitative questions:
\begin{itemize}
\item Chari and Kehoe show that limits on debt-to-value ratios and taxes on firm size can implement the sustainable efficient allocation.  Within banking, capital requirements limit debt-to-value ratios and a new part of the post-crisis regulations are additional capital surcharges for global systemically important banks. Of course, capital requirements existed before the 2008 financial crisis, but were raised following the financial crisis.  Do current capital requirements and capital surcharges for large bank implement the sustainably efficient outcome?  To my knowledge, most papers that focus on optimal capital requirements do not feature a government who cannot commit.
\item With quantitative model, you could decompose the size/degree of distortions caused by the governments inability to commit.
\end{itemize}
One simplification to improve computation time could be to limit to the contracts under consideration to be debt-equity contract and rely on the arguments that such contracts are optimal in Chari and Kehoe.

\textit{Spillovers and macroprudential regulations}---The government without commitment in Chari and Kehoe (2016) wants to bailout firms because bankruptcy is costly and ex post a bailout can help avoid the resource cost.  This notion of why a government would work to bailout a firm is microprudential. However, many of the bailouts during the 2008 financial crisis were motivated by more macroprudential concerns, meaning that the failure of a firm could cause the failure of other firms or indeed cause the entire system to breakdown---e.g., fire-sales.  For example, the motivation to bailout AIG was less due to concerns that the AIG bankruptcy would be costly in itself, but was because AIG underwrote a tremendous amount of CDS that was interconnected to many parts of the financial system and if AIG was left to fail it would have ripple effects throughout the entire financial system.  Chari and Kehoe specifically say in the introduction that they do not include spillovers to not muddy the analysis. However, exposure to spillovers seems something that could be additional private information to the manager.

\end{document}




