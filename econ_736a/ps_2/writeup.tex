\documentclass{article}
\usepackage{amsmath,amsthm,amssymb,amsfonts}
\usepackage{setspace,enumitem}
\usepackage{graphicx}
\usepackage{hyperref}
\usepackage{natbib}
\usepackage{afterpage}
\usepackage{xcolor}
\usepackage{etoolbox}
\usepackage{booktabs}
\usepackage{pdfpages}
\usepackage{bm}
\usepackage{multicol}
\usepackage{soul}
\usepackage{geometry}
\usepackage{accents}
\usepackage{accents}
\hypersetup{
	colorlinks,
	linkcolor={blue!90!black},
	citecolor={red!90!black},
	urlcolor={blue!90!black}
}
\usepackage{setspace}

\newtheorem{theorem}{Theorem}
\newtheorem{assumption}{Assumption}
\newtheorem{definition}{Definition}
\newtheorem{proposition}{Proposition}
\newtheorem{lemma}{Lemma}
\newcommand{\R}{\mathbb{R}}
\newcommand{\N}{\mathbb{N}}
\newcommand{\C}{\mathcal{C}}
\newcommand{\Lfn}{\mathcal{L}}
\newcommand{\Int}{\text{Int}}
\newcommand{\ubar}[1]{\underaccent{\bar}{#1}}
\newcommand{\xbf}{\mathbf{x}}
\newcommand{\Abf}{\mathbf{A}}
\newcommand{\Bbf}{\mathbf{B}}
\newcommand{\Gbf}{\mathbf{G}}
\newcommand{\bbf}{\mathbf{b}}

\usepackage{amsthm}

\newtheorem{manualtheoreminner}{Proposition}
\newenvironment{manualtheorem}[1]{%
  \renewcommand\themanualtheoreminner{#1}%
  \manualtheoreminner
}{\endmanualtheoreminner}

\title{ECON 736A: Problem Set 2}
\author{Alex von Hafften}
\date{\today}


\begin{document}

\maketitle

\section{Acharya and Dogra (2020)}

\begin{enumerate}

\item \textit{Carefully prove Proposition 1 in the paper}

\begin{manualtheorem}{1} 

Individual decision problem: Given a sequence of real interest rates, aggregate output, and idiosyncratic risk $\{r_t, y_t, \sigma_{y, t}\}$, household $i$'s consumption decision can be expressed as

\begin{align}
c_t^i = \C_t + \mu_t (a_t^i + y_t^i)
\end{align}

where $a_t^i = A_t^i / P_t$ is real net worth at the state of date $t$ and $\C_t$ and $\mu_t$ solve the following recursions:

\begin{align}
\C_t [1 + \mu_{t+1} (1 + r_t)] &= - \frac{1}{\gamma} \ln \beta (1 +r_t) + \C_{t+1} + \mu_{t+1} \bar y_{t+1} - \frac{\gamma \mu_{t+1}^2 \sigma_{y,t+1}^2}{2}\\
\mu_t &= \frac{\mu_{t+1}(1+r_t)}{1 + \mu_{t+1} (1+r_t)}
\end{align}

\end{manualtheorem}

\bigskip

We prove with guess and verify. With CARA utility, the household Euler equation is

$$
e^{-\gamma c_t^i} = \beta (1 + r_t) E_t e^{-\gamma c_{t+1}^i}
$$

Taking logs of both sides

$$
-\gamma c_t^i = \ln \beta (1 + r_t) + \ln E_t e^{-\gamma c_{t+1}^i}
$$

Guess consumption decision rule is 

$$
c_t^i = \C_t + \mu_t (a_t^i + y_t^i)
$$

where $\C_t$ and $\mu_t$ are deterministic. Using the HH budget constraint and the guessed consumption decision rule, we can get $c_{t+1}^i$:

\begin{align*}
a_{t+1}^i &= (1+r_t) (1 - \mu_t)(a_t^i + y_t^i) - (1 + r_t) \C_t\\
\implies
c_{t+1}^i 
&= \C_{t+1} + \mu_{t+1} [a_{t+1}^i + y_{t+1}^i]\\
&= \C_{t+1} + \mu_{t+1} [(1+r_t) (1 - \mu_t)(a_t^i + y_t^i) - (1 + r_t) \C_t + y_{t+1}^i]\\
\end{align*}

All terms in $c_{t+1}^i$ are known at $t$ except for $y_{t+1}^i$ which is normal, so $c_{t+1}^i$ is normal with

\begin{align*}
E_t[c_{t+1}^i] &= \C_{t+1} + \mu_{t+1} [(1+r_t) (1 - \mu_t)(a_t^i + y_t^i) - (1 + r_t) \C_t + \bar y_{t+1}]\\
Var_t[c_{t+1}^i] &= \mu_{t+1}^2 \sigma_{y,t+1}^2
\end{align*}

Applying MGF of normal distributions,

\begin{align*}
\ln E_t[e^{-\gamma c_{t+1}^i}] 
&= E_t[-\gamma c_{t+1}^i] + \frac{Var_t[-\gamma c_{t+1}^i]}{2}\\
&= -\gamma E_t[c_{t+1}^i] + \frac{\gamma^2Var_t[ c_{t+1}^i]}{2}\\
&= -\gamma \C_{t+1}  -\gamma  \mu_{t+1} [(1+r_t) (1 - \mu_t)(a_t^i + y_t^i) - (1 + r_t) \C_t + \bar y_{t+1}] + \frac{\gamma^2 \mu_{t+1}^2 \sigma_{y,t+1}^2}{2}
\end{align*}

Substituting into the logged Euler equation,

\begin{align*}
-\gamma [\C_t + \mu_t (a_t^i + y_t^i)] = \ln \beta (1 + r_t) -\gamma \C_{t+1}  -\gamma  \mu_{t+1} [(1+r_t) (1 - \mu_t)(a_t^i + y_t^i) - (1 + r_t) \C_t + \bar y_{t+1}] + \frac{\gamma^2 \mu_{t+1}^2 \sigma_{y,t+1}^2}{2}
\end{align*}

Matching coefficients on $(a_t^i + y_t^i)$:

\begin{align*}
-\gamma \mu_t &= -\gamma \mu_{t+1}(1+r_t)(1-\mu_t)\\
 \mu_t(1 + \mu_{t+1}(1+r_t)) &= \mu_{t+1}(1+r_t)\\
 \mu_t &= \frac{\mu_{t+1}(1+r_t)}{1 + \mu_{t+1}(1+r_t)}
\end{align*}

Matching constant coefficients:

\begin{align*}
-\gamma \C_t  &= \ln \beta (1 + r_t) -\gamma \C_{t+1}  +\gamma  \mu_{t+1} (1 + r_t) \C_t + \gamma\bar y_{t+1} + \frac{\gamma^2 \mu_{t+1}^2 \sigma_{y,t+1}^2}{2}\\
\C_t [1 + \mu_{t+1} (1 +r_t)] &= -\frac{1}{\gamma} \ln \beta (1+r_t) + \C_{t+1} + \mu_{t+1} \bar y_{t+1} - \frac{\gamma \mu_{t+1}^2 \sigma_{y, t+1}^2}{2} 
\end{align*}

\pagebreak

\item \textit{Carefully derive equations 4.1-4.4 in the paper.}

Linearizing (3.3), (3.5), (3.6), and (2.2)

\begin{align}
\mu_t &= \frac{\mu_{t+1}(1+r_t)}{1 + \mu_{t+1}(1+r_t)}\\
y_t &= y_{t+1} - \frac{\ln \beta (1+r_t)}{\gamma} - \frac{\gamma \mu_{t+1}^2 \sigma_{y,t+1}^2}{2} + g_t - g_{t+1}\\
\Psi \Pi_t(\Pi_t -1) &= 1 - \theta (1 - mc_t) + \Psi (\Pi_{t+1} - 1) \Pi_{t+1} \Bigg[ \frac{1}{1+r_t} \frac{x_{t+1}}{x_t} \Bigg]\\
1+i_t &= (1+r)\Pi_t^{\Phi_\pi} \ge 1
\end{align}

around the flexible price level of output $y^* = 1$, $\mu = \frac{r}{1+r}$, and $\Pi = 1$, we get

\begin{align}
\hat y_t &= \Theta \hat y_{t+1} - \frac{1}{\gamma} (i_t - \pi_{t+1}) - \Lambda \hat \mu_{t+1} \\ 
\hat \mu_t &= \tilde \beta \hat \mu_{t+1} + \tilde \beta (i_t - \pi_{t+1}) \\
\pi_t &= \tilde \beta \pi_{t+1} + \kappa \hat y_t\\
i_t &= \Phi_\pi \pi_t
\end{align}

where

\begin{align*}
\Theta &= 1 - \frac{\gamma \mu^2}{2} \frac{d \bm{\sigma}^2 (y^*)}{dy} \\
\Lambda &= \gamma \mu^2 \bm{\sigma}_y^2 ( y^*) \\
\tilde \beta &= \frac{1}{1 + r}
\end{align*}

$\kappa$ denotes the slope of the linearized Phillips curve, and $\hat y_t$, $\hat \mu_t$, $i_t$, and $\pi_t$ denote the log deviation of $y_t$, $\mu_t$, $1+i_t$, and $\Pi_t$ from their steady state values.

\end{enumerate}

\pagebreak

\section{Werning (2015)}

\begin{enumerate}

\item \textit{Carefully derive Proposition 5 in the paper.}

\begin{manualtheorem}{5} 
Suppose utilities satisfy are log utility, household income satisfies $\gamma_t^i (s, Y) = \tilde \gamma_t^i(s) Y$ (i.e., proportional to aggregate income) and borrowing constraints satisfy $B_t^i(s, Y) = \tilde B_t^i (s) Y$ (i.e., also proportional to aggregate income. In addition, suppose initial bond holdings are zero $b^i_0 = 0$ for
all households. Then $\{C_t, R_t\}$ is part of an equilibrium if and only if

\begin{align}
U'(C_t) = \beta_t R_t U'(C_{t+1}) \label{w_eq}
\end{align}

for some sequence of discount factors $\{\beta_t\}$, independent of both $\{R_t\}$ and $\{C_t\}$.

\end{manualtheorem}

\bigskip

I follow Werning's discussion following the proposition.

First, we can construct the discount factors $\beta_t$ and household allocation by considering a ``reference" equilibrium where aggregate income is constant $\tilde Y_t = 1$. This equilibrium includes interest rates $\{\tilde R_t\}$, household consumption and wealth $\{\tilde c(s^t;a_0), \tilde a(s^t; a_0)\}$, and asset prices are the discounted value of future dividends $\tilde q_t = \sum_{s=0}^\infty (\tilde R_t \tilde R_{t+1} ... \tilde R_{t+1})^{-1} \tilde d_{t+1+s}$. Define $\beta_t \equiv \frac{1}{\tilde R_t}$.  The aggregate euler equation (\ref{w_eq}) holds trivially:

\begin{align*}
U'(\tilde C_t) &= \beta_t \tilde R_t U'(\tilde C_{t+1}) \\
\iff
U'(\tilde Y_t) &= \frac{1}{\tilde R_t} \tilde R_t U'(\tilde Y_{t+1}) \\
\iff
U'(1) &= U'(1)
\end{align*}

We can now consider another sequence $\{C_t, R_t\}$ that satisfies (\ref{w_eq}) and guess the equilibrium objects and verify that equilibrium conditions hold. We can guess that household $i$ consumption and wealth, interest rates, and asset prices are the following:

\begin{align*}
c^i(s^t; a_0) &\equiv \tilde c^i(s^t; a_0) C_t\\
a^i(s^t; a_0) &\equiv \tilde a^i(s^t; a_0) C_t\\
R_t &\equiv \tilde R_t \frac{C_{t+1}}{C_t}\\
q_t &\equiv \tilde q_t C_t
\end{align*}

For the household Euler equation, start with the household Euler equation in the reference equilibrium:

\begin{align*}
U'(\tilde c^i(s^t; a_0)) &\ge \beta \tilde R_t E_t[U'(\tilde c^i(s^{t+1}; a_0))]\\
\iff
U'(\tilde c^i(s^t; a_0))\frac{C_t}{C_t} &\ge \beta \tilde R_t E_t\Bigg[U'(\tilde c^i(s^{t+1}; a_0))\frac{C_{t+1}}{C_{t+1}}\Bigg]\\
\iff
U'( c^i(s^t; a_0)) &\ge \beta \frac{\tilde R_t C_{t+1}}{C_t} E_t\Bigg[U'( c^i(s^{t+1}; a_0))\Bigg]\\
\iff
U'( c^i(s^t; a_0)) &\ge \beta  R_t E_t [U'( c^i(s^{t+1}; a_0))]
\end{align*}


\end{enumerate}

\end{document}




