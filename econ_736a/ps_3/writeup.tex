\documentclass{article}
\usepackage{amsmath,amsthm,amssymb,amsfonts}
\usepackage{setspace,enumitem}
\usepackage{graphicx}
\usepackage{hyperref}
\usepackage{natbib}
\usepackage{afterpage}
\usepackage{xcolor}
\usepackage{etoolbox}
\usepackage{booktabs}
\usepackage{pdfpages}
\usepackage{bm}
\usepackage{multicol}
\usepackage{soul}
\usepackage{geometry}
\usepackage{accents}
\usepackage{accents}
\hypersetup{
	colorlinks,
	linkcolor={blue!90!black},
	citecolor={red!90!black},
	urlcolor={blue!90!black}
}
\usepackage{setspace}

\newtheorem{theorem}{Theorem}
\newtheorem{assumption}{Assumption}
\newtheorem{definition}{Definition}
\newtheorem{proposition}{Proposition}
\newtheorem{lemma}{Lemma}
\newcommand{\R}{\mathbb{R}}
\newcommand{\N}{\mathbb{N}}
\newcommand{\C}{\mathcal{C}}
\newcommand{\Lfn}{\mathcal{L}}
\newcommand{\Int}{\text{Int}}
\newcommand{\ubar}[1]{\underaccent{\bar}{#1}}
\newcommand{\xbf}{\mathbf{x}}
\newcommand{\Abf}{\mathbf{A}}
\newcommand{\Bbf}{\mathbf{B}}
\newcommand{\Gbf}{\mathbf{G}}
\newcommand{\bbf}{\mathbf{b}}

\usepackage{amsthm}

\newtheorem{manualtheoreminner}{Proposition}
\newenvironment{manualtheorem}[1]{%
  \renewcommand\themanualtheoreminner{#1}%
  \manualtheoreminner
}{\endmanualtheoreminner}

\title{ECON 736A: Problem Set 3}
\author{Alex von Hafften}
\date{\today}


\begin{document}

\maketitle

\section{Atkeson, Chari, and Kehoe (2007)}

\begin{enumerate}

\item \textit{Prove Propositions 2 and 3.}


\begin{manualtheorem}{2}[Determinacy of Equilibrium and the Taylor Principle]
The linear equilibria with interest rate rules of the Taylor rule form $i_t = \bar \iota + a x_t$ have outcomes
of the form
\begin{align*}
x_{t+1} &= i_t + c\eta_t\\
\pi_t &= x_t + (1 + \sigma c)\eta_t\\
y_t &= (1 + \sigma c)\eta_t
\end{align*}
For every $a < 1$ and $\bar i$, the economy has a continuum of equilibria indexed by the parameter
$c$. For every $a \ge 1$ and $\bar i$, within the class of bounded linear equilibria, the economy has
$a$ unique equilibrium with $c = 0$.
\end{manualtheorem}

\bigskip

\textbf{Proof:} First, verify that the candidate equilibrium satisfies the equilibrium conditions in the lemma:
\begin{align*}
E[y_t|h_{t-1}] 
&= E[(1+\sigma c)\eta_t|h_{t-1}]\\
&= (1+\sigma c)E[\eta_t|h_{t-1}]\\
&= 0\\
E[\pi_{t+1}|h_{t-1}] 
&= E[ x_{t+1} + (1 + \sigma c)\eta_{t+1}|h_{t-1}]\\
&= E[ i_t + c\eta_t|h_{t-1}] + (1 + \sigma c) E[\eta_{t+1}|h_{t-1}]\\
&= i_t + cE[ \eta_t|h_{t-1}]\\
&= i_t \\
\end{align*}
Second, verify that the candidate equilibrium satisfies the Phillips curve and the Euler equation:
\begin{align*}
y_t &= \pi_t - x_t\\
\implies
[(1+\sigma c)\eta_t] &= [x_t + (1+\sigma c)\eta_t] - x_t\\
\implies
(1+\sigma c)\eta_t &= (1+\sigma c)\eta_t\\
y_t &= E[y_{t+1}|h_t] - \sigma ( i_t - E[\pi_{t+1}|h_t]) + \eta_t \\
\implies 
(1+\sigma c) \eta_t 
&= - \sigma ( i_t - E[\pi_{t+1}|h_t]) + \eta_t\\
&= - \sigma ( i_t - E[x_{t+1} + (1+\sigma c) \eta_{t+1}|h_t]) + \eta_t\\
&= - \sigma ( i_t - E[x_{t+1}|h_t]) + \eta_t \\
&= - \sigma ( i_t - E[i_t + c \eta_t|h_t]) + \eta_t \\
&= - \sigma ( -c \eta_t) + \eta_t \\
&= (1+\sigma c) \eta_t 
\end{align*}
Finally, we can derive a first order difference equation for $x$ using lemma, the Taylor rule, and agent optimize and representativeness $x_t(h_{t-1}) = z_t(h_{t-1}) = E[\pi_t|h_{t-1}]$:
\begin{align*}
E[\pi_{t+1}|h_{t-1}] 
&= i_t\\
&= \bar \iota + a x_t(h_{t-1})\\
\implies 
x_{t+1}(h_t) 
&= \bar \iota + a x_t(h_{t-1})
\end{align*}
A solution to the first order difference equation for $x$ is
\begin{align*}
x_t = a^{(t-s)} \Bigg(x_s - \frac{\bar \iota}{1-a}\Bigg) + \frac{\bar \iota}{1-a}
\end{align*}
for some $s < t$. For an equilibrium to exist, $x_t$ must not diverge. If $a < 1$, $x_t \to \frac{\bar \iota}{1-a}$, so for any $c$, an equilibrium exists. If $a > 1$, for $x_t$ not to diverge, the following must hold:
\begin{align*}
x_s - \frac{\bar\iota}{1-a} &= 0
\end{align*}
for all $s$. Thus, $x_{s+1} = x_s$, so
\begin{align*}
x_{s+1}  
&= \bar\iota + ax_{s+1} \\
&= \bar\iota + ax_{s}\\
&= i_s\\
\implies
c &= 0
\end{align*}

\pagebreak

\begin{manualtheorem}{3}[Rules Satisfying the Taylor Principle are Inefficient]
The outcomes under a Taylor rule of the form $i_t = \bar\iota + a x_t$ with $a > 1$ are dominated by the outcomes of an equilibrium with $a = 0$ and $\bar \iota = 0$.
\end{manualtheorem}

\bigskip

\textbf{Proof:} By Proposition 2, the outcome under the Taylor rule of the form $i_t = \bar\iota + a x_t$ with $a > 1$ is unique:

\begin{align*}
x_{t+1} &= i_t = \frac{\bar\iota}{1-a}\\
\pi_t &= x_t + \eta_t\\
y_t &= \eta_t
\end{align*}

Thus, representative agent expected payoff is

\begin{align*}
E\Bigg[r^A\Bigg(\frac{\bar\iota}{1-a}, \frac{\bar\iota}{1-a}, \frac{\bar\iota}{1-a} + \eta_t\Bigg)\Bigg] 
&=
-\frac{1}{2} E\Bigg[ \Bigg(\frac{\bar\iota}{1-a} - \Bigg(\frac{\bar\iota}{1-a} + \eta_t\Bigg)\Bigg)^2 + (\eta_t - \bar y)^2 + \Bigg(\frac{\bar\iota}{1-a} + \eta_t\Bigg)^2\Bigg]\\
&=
-\frac{1}{2} E\Bigg[ \eta_t^2 + (\eta_t^2 - 2\bar y\eta_t + \bar y^2) + \Bigg(\frac{\bar\iota}{1-a}\Bigg)^2 + 2 \frac{\bar\iota}{1-a} \eta_t+ \eta_t^2)\Bigg]\\
&=
-\frac{1}{2}\Bigg[ 2E[ \eta_t^2] - 2\bar yE[\eta_t] + \bar y^2 + \Bigg(\frac{\bar\iota}{1-a}\Bigg)^2 + 2 \Bigg(\frac{\bar\iota}{1-a}\Bigg) E[\eta_t]+ E[\eta_t^2] \Bigg]\\
&=
-\frac{1}{2}\Bigg[ 3\sigma_\eta^2+ \bar y^2 + \Bigg(\frac{\bar\iota}{1-a}\Bigg)^2\Bigg]
\end{align*}

If $\bar \iota = a = 0$, then $i_t = 0$, but the equilibrium is not unique and for every $c$, the following is an equilibrium. Thus, there is an equilibrium for $c = 0$ where

\begin{align*}
x_{t+1} &= 0\\
\pi_t &= \eta_t\\
y_t &=\eta_t
\end{align*}

And the representative agent expected payoff in that equilibrium is:

\begin{align*}
E[r^A(0, 0,  \eta_t)]
&=
-\frac{1}{2} E[ \eta_t^2 + (\eta_t - \bar y)^2 + \eta_t^2]\\
&=
-\frac{1}{2} \Bigg[E[ \eta_t^2] + E[\eta_t^2] - 2\bar yE[\eta_t] + \bar y^2 + E[\eta_t^2]\Bigg]\\
&=
-\frac{1}{2} \Bigg[3 \sigma_\eta^2 + \bar y^2 \Bigg]
\end{align*}

The equilibrium associated Taylor principle is not optimal, if there exists a $c$ such that the representative agent is better off:

\begin{align*}
E[r^A(0, 0,  \eta_t)] &> E\Bigg[r^A\Bigg(\frac{\bar\iota}{1-a}, \frac{\bar\iota}{1-a}, \frac{\bar\iota}{1-a} + \eta_t\Bigg)\Bigg] \\
\iff
-\frac{1}{2} \Bigg[3 \sigma_\eta^2 + \bar y^2 \Bigg] &> -\frac{1}{2}\Bigg[ 3\sigma_\eta^2+ \bar y^2 + \Bigg(\frac{\bar\iota}{1-a}\Bigg)^2\Bigg]\\
\iff 0  &<  \Bigg(\frac{\bar\iota}{1-a}\Bigg)^2
\end{align*}

Which holds for $a > 1$ and $\bar\iota > 0$.

\pagebreak

\item \textit{Consider a slight extension of the environment we saw in class.
There, we assumed that the agents did not observe the money growth
rate but only observed the inflation rate (which was a noisy signal of
the money growth rate). Now suppose that in addition to observing the
inflation, agents observe the true money growth rate choice but with a
lag. In particular, suppose that $\mu_{t-1}$ is observable in period $t$ after wages
are chosen. Is transparency valuable in this case? More specifically, prove
the analog of Proposition 6 in the paper for this environment.}

\end{enumerate}

\end{document}




