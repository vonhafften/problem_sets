\documentclass[]{article}
\usepackage{lmodern}
\usepackage{amssymb,amsmath}
\usepackage{ifxetex,ifluatex}
\usepackage{fixltx2e} % provides \textsubscript
\ifnum 0\ifxetex 1\fi\ifluatex 1\fi=0 % if pdftex
  \usepackage[T1]{fontenc}
  \usepackage[utf8]{inputenc}
\else % if luatex or xelatex
  \ifxetex
    \usepackage{mathspec}
  \else
    \usepackage{fontspec}
  \fi
  \defaultfontfeatures{Ligatures=TeX,Scale=MatchLowercase}
\fi
% use upquote if available, for straight quotes in verbatim environments
\IfFileExists{upquote.sty}{\usepackage{upquote}}{}
% use microtype if available
\IfFileExists{microtype.sty}{%
\usepackage[]{microtype}
\UseMicrotypeSet[protrusion]{basicmath} % disable protrusion for tt fonts
}{}
\PassOptionsToPackage{hyphens}{url} % url is loaded by hyperref
\usepackage[unicode=true]{hyperref}
\hypersetup{
            pdftitle={ECON 712 - PS 5},
            pdfauthor={Alex von Hafften},
            pdfborder={0 0 0},
            breaklinks=true}
\urlstyle{same}  % don't use monospace font for urls
\usepackage[margin=1in]{geometry}
\usepackage{graphicx,grffile}
\makeatletter
\def\maxwidth{\ifdim\Gin@nat@width>\linewidth\linewidth\else\Gin@nat@width\fi}
\def\maxheight{\ifdim\Gin@nat@height>\textheight\textheight\else\Gin@nat@height\fi}
\makeatother
% Scale images if necessary, so that they will not overflow the page
% margins by default, and it is still possible to overwrite the defaults
% using explicit options in \includegraphics[width, height, ...]{}
\setkeys{Gin}{width=\maxwidth,height=\maxheight,keepaspectratio}
\IfFileExists{parskip.sty}{%
\usepackage{parskip}
}{% else
\setlength{\parindent}{0pt}
\setlength{\parskip}{6pt plus 2pt minus 1pt}
}
\setlength{\emergencystretch}{3em}  % prevent overfull lines
\providecommand{\tightlist}{%
  \setlength{\itemsep}{0pt}\setlength{\parskip}{0pt}}
\setcounter{secnumdepth}{0}
% Redefines (sub)paragraphs to behave more like sections
\ifx\paragraph\undefined\else
\let\oldparagraph\paragraph
\renewcommand{\paragraph}[1]{\oldparagraph{#1}\mbox{}}
\fi
\ifx\subparagraph\undefined\else
\let\oldsubparagraph\subparagraph
\renewcommand{\subparagraph}[1]{\oldsubparagraph{#1}\mbox{}}
\fi

% set default figure placement to htbp
\makeatletter
\def\fps@figure{htbp}
\makeatother

\newcommand{\N}{\mathbb{N}}
\newcommand{\Z}{\mathbb{Z}}
\newcommand{\R}{\mathbb{R}}
\newcommand{\Q}{\mathbb{Q}}
\usepackage{bm}

\title{ECON 712 - PS 5}
\author{Alex von Hafften\footnote{I worked on this problem set with a study
  group of Michael Nattinger, Andrew Smith, and Ryan Mather. I also
  discussed problems with Emily Case, Sarah Bass, and Danny Edgel.}}
\date{10/1/2020}

\begin{document}
\maketitle

In this problem set we study the macroeconomic consequences of
eliminating the Social Security system in the U.S. To do so, we set up
and solve a simple general equilibrium overlapping generations model.
This model is a simplified version of the model by Conesa and Krueger
(1999). You are not required to write the code from scratch to solve
this model, we provide you with already written code with some missing
parts. You are asked to understand the logic of the code, complete
missing parts and use it to run policy experiments.

\section{1 Model}\label{model}

Consider the model presented during Friday's discussion section.

Each period a continuum of agents is born. Agents live for \(J\) periods
after which they die. The population growth rate is \(n\) per year
(which is the model period length). Thus, the relative size of each
cohort of age \(j\), \(\psi_j\), is given by:

\[\psi_{i+1} = \frac{\psi_i}{1+n}\]

for \(i=1, ..., J - 1\) with \(\psi_1 = \bar{\psi} > 0\). It is
convenient to normalize \(\psi\), so that it sums up to 1 across all age
groups.

Newly born agents (i.e. \(j=1\)) are endowed with no initial capital
(i.e., \(k_j=0\)) but can subsequently save in capital which they can
rent to firms at rate \(r\). A worker of age \(j\) supplies labor
\(\ell_j \in [0,1]\) and pays proportional social security taxes on her
labor income \(\tau w e_j \ell_j\) until she retires at age \(J^R < J\),
where \(e_j\) is the age-efficiency profile. Upon retirement, agent
receives pension benefits \(b\).

The instantaneous utility function of a worker at age
\(j=1, 2, ..., J^R-1\) is given by:

\[u^{W} (c_j, \ell_j) = \frac{(c_j^{\gamma}(1-\ell_j)^{1-\gamma})^{1-\sigma}}{1 - \sigma}\]

with \(c_j\) denoting consumption and \(\ell_j\) denoting labor supply
at age \(j\). The weight on consumption is \(\gamma\) and the
coefficient of relative risk aversion is \(\sigma\). The instantaneous
utility function of a retired agent at age \(j = J^R, ..., J\) is given
by:

\[u^R(c_j) = \frac{c_j^{1-\sigma}}{1-\sigma}.\]

\pagebreak

Preferences are then given by

\[\sum_{j=1}^{J^R-1}\beta^{j-1}u^W(c_j, \ell_j) + \sum_{j=J^R}^J \beta^{j-1}u^R(c_j)\]

There is a constant returns to scale production technology
\(Y=F(K, L)=K^\alpha L^{1-\alpha}\) with \(\alpha\) denoting capital
share, \(Y\) denoting aggregate output, \(K\) denoting aggregate capital
stock, and \(L\) denoting aggregate effective labor supply. The capital
depreciates at rate \(\delta\). Capital and labor markets are perfectly
competitive.

\subsection{1.1 Parametrization}\label{parametrization}

\begin{align*}
J &= 66 \\
J^R &= 46 \\
n &=  0.011  \\
k_1 &= 0 \\
\tau &= 0.11 \\
\gamma &= 0.42 \\
\sigma &= 2 \\
\beta &= 0.97  \\
\alpha &= 0.36 \\
\delta &= 0.06 \\
\end{align*}

\section{2 Questions}\label{questions}

\subsection{2.1}\label{section}

Derive the below equation for labor supply, used in the solution of
workers' recursive problem (refer to lecture notes for details).

\[
\ell = \frac{\gamma(1-\tau)e_jw-(1 - \gamma)[(1+r)k-k']}{(1-\tau)e_jw}
\]

The dynamic programming problem of a \(j\)-year-old worker,
\(j=1, ..., J^R-1\), is given by (equation (8) from the section
handout):

\begin{align*}
V_j(k) = \max_{k', \ell} & \{u^{W}((1 - \tau)we_j\ell+(1+r)k-k', \ell) + \beta V_{j+1}(k')\}\\
\text{s.t. } & k = 0 \text{ if } j = 1\\
\text{and } & 0 \le \ell \le 1\\
\end{align*}

Taking the FOC of \(V_j\) with respect to \(\ell\):

\[
\frac{\partial V_j}{\partial \ell} = \frac{\partial}{\partial \ell}\Big(u^W((1 - \tau)we_j\ell+(1+r)k-k', \ell)\Big) + \beta (0)\\
 = \frac{\partial}{\partial \ell}\Big(u^W((1 - \tau)we_j\ell+(1+r)k-k', \ell)\Big) \\
 = \frac{\partial u^W}{\partial \ell}\Big(c(\ell), \ell)\Big)\frac{\partial c}{\partial \ell} (\ell)
\]

Where \(c(\ell)=(1 - \tau)we_j\ell+(1+r)k-k'\).

\[
\frac{\partial c}{\partial \ell}(\ell) = (1 - \tau)we_j
\]

The FOC of \(u^W\) with respect to \(\ell\) is:

\$\$ \frac{\partial u^W}{\partial \ell} (c(\ell), \ell) = (1 -
\sigma)\frac{(c(\ell)^{\gamma}(1-\ell)^{1-\gamma})^{-\sigma}}{1 - \sigma}
\Big[(1-\gamma)c(\ell)^{\gamma}(1-\ell)^{-\gamma}(-1) + \gamma c(\ell)^{\gamma-1}(1-\ell)^{1-\gamma}\frac{\partial c}{\partial \ell} (\ell) \Big] \textbackslash{}
=
\frac{(1-\gamma)c(\ell)^{\gamma}(1-\ell)^{-\gamma}(-1) + \gamma c(\ell)^{\gamma-1}(1-\ell)^{1-\gamma}\frac{\partial c}{\partial \ell} (\ell)}{(c(\ell)^{\gamma}(1-\ell)^{1-\gamma})^{-\sigma} }
\textbackslash{}

\$\$

\end{document}
