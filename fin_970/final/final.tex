\documentclass{article}
\usepackage{amsmath,amsthm,amssymb,amsfonts}
\usepackage{setspace,enumitem}
\usepackage{graphicx}
\usepackage{hyperref}
\usepackage{natbib}
\usepackage{afterpage}
\usepackage{xcolor}
\usepackage{etoolbox}
\usepackage{booktabs}
\usepackage{pdfpages}
\usepackage{multicol}
\usepackage{geometry}
\usepackage{accents}
\usepackage{bbm}
\usepackage{placeins}
\usepackage{verbatim}
\hypersetup{
	colorlinks,
	linkcolor={blue!90!black},
	citecolor={red!90!black},
	urlcolor={blue!90!black}
}

\newtheorem{theorem}{Theorem}
\newtheorem{assumption}{Assumption}
\newtheorem{definition}{Definition}
\newtheorem{lemma}{Lemma}
\setlength{\parindent}{0cm}
\geometry{margin = 1in}

\newcommand{\R}{\mathbb{R}}
\newcommand{\ubar}[1]{\underaccent{\bar}{#1}}
\newcommand{\Int}{\text{Int}}
\newcommand{\xbf}{\mathbf{x}}
\newcommand{\Abf}{\mathbf{A}}
\newcommand{\Bbf}{\mathbf{B}}
\newcommand{\Gbf}{\mathbf{G}}
\newcommand{\bbf}{\mathbf{b}}
\newcommand{\one}{\mathbbm{1}}

\newtoggle{extended}
\settoggle{extended}{false}

\title{FIN 970: Final Exam}
\author{Alex von Hafften}

\begin{document}

\maketitle

\section{Problem 1a: Term Structure and No Arbitrage Models}

\begin{enumerate}

\item Use SDF approach.

\textbf{Solution:} Conjecture $P_t^n = \exp(A_n + B_n'X_t)$. Proof by induction.

For $n=1$,

\begin{align*}
P_t^1 
&= 
E_t[M_{t+1} \cdot 1]\\
\implies\exp(A_1 + B_1'X_t) 
&= 
E_t \Bigg[ \exp \Bigg(-\delta_0 - \delta_1 X_t - \frac{1}{2} \lambda_t'\lambda_t - \lambda_t' \varepsilon_{t+1}\Bigg) \Bigg]\\
\implies E_t \Bigg[ -\delta_0 - \delta_1 X_t - \frac{1}{2} \lambda_t'\lambda_t - \lambda_t' \varepsilon_{t+1} \Bigg]
&= -\delta_0 -\delta_1 X_t - \frac{1}{2}\lambda_t'\lambda_t \\
Var_t\Bigg[ -\delta_0 - \delta_1 X_t - \frac{1}{2} \lambda_t'\lambda_t - \lambda_t' \varepsilon_{t+1} \Bigg]
&= \lambda_t'\lambda_t\\
\implies
\exp(A_1 + B_1'X_t) 
&= 
\exp (-\delta_0 - \delta_1 X_t ) \\
\implies
\begin{cases}
A_1 = -\delta_0\\
B_1 = -\delta_1'
\end{cases}
\end{align*}

For some $n>1$, the Euler equation holds:

\begin{align*}
P_t^n &= E_t[M_{t+1} P_{t+1}^{n-1}]\\
\exp(A_{n} + B_{n}'X_{t})
&= E_t \Bigg[\exp \Bigg(-r_t - \frac{1}{2} \lambda_t'\lambda_t - \lambda_t' \varepsilon_{t+1}\Bigg) \exp(A_{n-1} + B_{n-1}'X_{t+1})\Bigg]\\
&= E_t \Bigg[\exp \Bigg(-\delta_0 - \delta_1 X_t - \frac{1}{2} \lambda_t'\lambda_t - \lambda_t' \varepsilon_{t+1} + A_{n-1} + B_{n-1}'(\mu + \Phi X_t + \Sigma \varepsilon_{t+1} ) \Bigg)\Bigg]\\
&= E_t \Bigg[\exp \Bigg(-\delta_0 - \delta_1 X_t - \frac{1}{2} \lambda_t'\lambda_t  + A_{n-1} + B_{n-1}'\mu + B_{n-1}'\Phi X_t + [B_{n-1}'\Sigma - \lambda_t'] \varepsilon_{t+1}  \Bigg)\Bigg]
\end{align*}

\begin{align*}
& E_t \Bigg[-\delta_0 - \delta_1 X_t - \frac{1}{2} \lambda_t'\lambda_t  + A_{n-1} + B_{n-1}'\mu + B_{n-1}'\Phi X_t + [B_{n-1}'\Sigma - \lambda_t'] \varepsilon_{t+1}  \Bigg] \\
&=  -\delta_0 - \delta_1 X_t - \frac{1}{2} \lambda_t'\lambda_t  + A_{n-1} + B_{n-1}'\mu + B_{n-1}'\Phi X_t\\
& Var_t\Bigg[-\delta_0 - \delta_1 X_t - \frac{1}{2} \lambda_t'\lambda_t  + A_{n-1} + B_{n-1}'\mu + B_{n-1}'\Phi X_t + [B_{n-1}'\Sigma - \lambda_t'] \varepsilon_{t+1}  \Bigg] \\
&= [B_{n-1}'\Sigma - \lambda_t'][B_{n-1}'\Sigma - \lambda_t']'\\
&= B_{n-1}'\Sigma \Sigma' B_{n-1} + \lambda_t' \lambda_t - 2 B_{n-1}'\Sigma \lambda_t
\end{align*}

\begin{align*}
\exp(A_{n} + B_{n}'X_{t})
&= \exp\Bigg(-\delta_0 - \delta_1 X_t - \frac{1}{2} \lambda_t'\lambda_t  + A_{n-1} + B_{n-1}'\mu + B_{n-1}'\Phi X_t \\&+ \frac{1}{2} B_{n-1}'\Sigma \Sigma' B_{n-1} + \frac{1}{2}\lambda_t' \lambda_t - B_n'\Sigma (\lambda_0 + \lambda_1 X_t) \Bigg)\\
&= \exp\Bigg(-\delta_0   + A_{n-1} + B_{n-1}'\mu - B_{n-1}'\Sigma \lambda_0 + \frac{1}{2} B_{n-1}'\Sigma \Sigma' B_{n-1}+ (- \delta_1  + B_{n-1}'\Phi - B_{n-1}'\Sigma \lambda_1 )X_t  ) \Bigg)\\
\implies
&
\begin{cases}
A_n = - \delta_0 + A_{n-1} + B_{n-1}'(\mu - \Sigma \lambda_0) + \frac{1}{2} B_{n-1}'\Sigma \Sigma' B_{n-1}\\
B_n = - \delta_1  + (\Phi - \Sigma \lambda_1)' B_{n-1}
\end{cases}
\end{align*}

\item Use risk-neutral density approach.

\textbf{Solution:} Conjecture $P_t^n = \exp(C_n + D_n'X_t)$. Proof by induction.

For $n=0$,

\begin{align*}
P_t^1 
&= 
e^{-r_t}E_t^Q[1]\\
\exp(C_1 + D_1'X_t) 
&= 
\exp (-\delta_0 - \delta_1 X_t) \\
\implies&
\begin{cases}
C_1 = -\delta_0\\
D_1 = -\delta_1'
\end{cases}
\end{align*}

For some $n>1$, the Euler equation holds:

\begin{align*}
P_t^n 
&= e^{-r_t} E_t^Q [P_{t+1}^{n-1}]\\
\exp(C_n + D_n'X_t)
&= e^{-r_t} E_t^Q [\exp(C_{n-1} + D_{n-1}'X_{t+1})]\\
&= e^{-r_t} E_t^Q [\exp(C_{n-1} + D_{n-1}'(\mu^Q + \Phi^Q X_t + \Sigma \varepsilon_{t+1}^Q))]\\
&= e^{-r_t} E_t^Q [\exp(C_{n-1} + D_{n-1}'\mu^Q + D_{n-1}'\Phi^Q X_t + D_{n-1}'\Sigma \varepsilon_{t+1}^Q)]
\end{align*}

\begin{align*}
E_t^Q[C_{n-1} + D_{n-1}'\mu^Q + D_{n-1}'\Phi^Q X_t + D_{n-1}'\Sigma \varepsilon_{t+1}^Q] 
&= E_t[C_{n-1} + D_{n-1}'\mu^Q + D_{n-1}'\Phi^Q X_t]\\
Var_t^Q [C_{n-1} + D_{n-1}'\mu^Q + D_{n-1}'\Phi^Q X_t + D_{n-1}'\Sigma \varepsilon_{t+1}^Q]
&= D'_{n-1} \Sigma \Sigma' D_{n-1}
\end{align*}

\begin{align*}
\exp(C_n + D_n'X_t)
&=  \exp \Bigg(-\delta_0 - \delta_1 X_t + C_{n-1} + D_{n-1}'\mu^Q + D_{n-1}'\Phi^Q X_t + \frac{1}{2} D'_{n-1} \Sigma \Sigma' D_{n-1} \Bigg)\\
& \begin{cases}
C_n = -\delta_0  + C_{n-1} + D_{n-1}'\mu^Q  + \frac{1}{2} D'_{n-1} \Sigma \Sigma' D_{n-1}\\
D_n = - \delta_1' + \Phi^{Q'} D_{n-1}
\end{cases}
\end{align*}

\item Show that this is a one-to-one mapping between the risk-neutral parameters $(\mu^Q,  \Phi^Q)$ and the market prices of risk $(\lambda_0, \lambda_1)$.

\bigskip

\textbf{Solution:} Clearly, parts (1) and (2) are equivalent iff

\begin{align*}
\mu^Q &= \mu - \Sigma \lambda_0  \iff \lambda_0 = \Sigma^{-1}(\mu - \mu^Q)\\
\Phi^Q &= \Phi - \Sigma \lambda_1 \iff \lambda_1 = \Sigma^{-1}(\Phi - \Phi^Q)
\end{align*}

with $A_n = C_n$ and $B_n = D_n$. Thus, we can go back and forth from SDF to risk-neutral densities to price bonds of any maturity.

\end{enumerate}

\pagebreak

\section{Problem 2h}

\begin{enumerate}

\item Conjecture $pc_t$ is linear in state variables. Solve for $pc_t$ and $r_{c,t+1}$. Explain how risk exposures depend on preferences and consumption dynamics.

\bigskip

\textbf{Solutions:} Conjecture that $pc_t = A_0 + A_x x_t$. 
Using the Campbell-Schiller approximation to log-linearization the consumption return:

\begin{align*}
R_{C,t+1} 
&= \frac{P_{C,t+1} + C_{t+1}}{P_{C,t}} \\
&=  \frac{\frac{P_{C,t+1}}{C_{t+1}} + 1}{\frac{P_{C,t}}{C_t}} \frac{C_{t+1}}{C_{t}}\\
r_{c,t+1} 
&= \log(\exp(pc_{t+1}) +1) - pc_t +\Delta c_{t+1} \\
&\approx \Bigg[\log(\exp(\bar{pc}) +1) + \frac{\exp(\bar{pc})}{\exp(\bar{pc}) + 1} (pc_{t+1} - \bar{pc})\Bigg]- pc_t +\Delta c_{t+1} \\
&= \underbrace{\log(\exp(\bar{pc}) +1)- \frac{\exp(\bar{pc})}{\exp(\bar{pc}) + 1} \bar{pc}}_{\equiv \kappa_0} + \underbrace{\frac{\exp(\bar{pc})}{\exp(\bar{pc}) + 1}}_{\equiv \kappa_1} pc_{t+1} - pc_t +\Delta c_{t+1} 
\end{align*}

where $pc_t = \log(P_{C,t}/C_{t})$. Alternatively, we can express the return in terms of the demeaned price-consumption ratio,

\begin{align*}
r_{c, t+1} 
&= \kappa_0 + \kappa_1 pc_{t+1} - pc_t + \Delta c_{t+1}\\
&= -\log \kappa_1 + \kappa_1 \underbrace{\tilde{pc}_{t+1}}_{\equiv pc_{t+1} - \bar{pc}} - \tilde{pc}_t + \Delta c_{t+1}
\end{align*}

Given the guess for $pc_t$, its unconditional expected value is $\bar{pc} = A_0$, so $\tilde{pc}_t = A_x x_t$. Plugging the dynamics for volatility and consumption:

\begin{align*}
\tilde{pc}_{t+1} 
&= A_x x_{t+1} \\
&= A_x [\rho x_t + \varphi_e \sigma e_{t+1}] \\
&= A_x \rho x_t  + A_x \varphi_e \sigma e_{t+1}  
\end{align*}

Plugging into the consumption return:

\begin{align*}
r_{c, t+1} 
&= -\log \kappa_1 + \kappa_1 \tilde{pc}_{t+1} - \tilde{pc}_t + \Delta c_{t+1}\\
&= -\log \kappa_1 + \kappa_1 [A_x \rho x_t  + A_x \varphi_e \sigma e_{t+1} ] - [A_x x_t] + [\mu + x_t + \sigma \varepsilon_{t+1}] \\
&= [-\log \kappa_1 + \mu] +  [\kappa_1 A_x \rho - A_x + 1] x_t  + \kappa_1 A_x \varphi_e \sigma e_{t+1} + \sigma \varepsilon_{t+1}
\end{align*}


For any asset with return $R_{i,t+1}$, the Euler equation holds and if the return is log-normal:

\begin{align*}
1 &= E_t[M_{t+1} R_{i,t+1}]\\
&= E_t[\exp(m_{t+1} + r_{i,t+1})]\\
&= \exp(E_t[m_{t+1} + r_{i,t+1}] + \frac{1}{2} Var_t[m_{t+1} + r_{i,t+1}])\\
\implies
0&= E_t[m_{t+1} + r_{i,t+1}] + \frac{1}{2} Var_t[m_{t+1} + r_{i,t+1}]
\end{align*}


In particular for the consumption assets, the Euler equation holds. 

\begin{align*}
m_{t+1} + r_{c,t+1} 
&= \theta \log \delta - \frac{\theta}{\psi} \Delta c_{t+1} + \theta r_{c,t+1}\\
&= \theta \log \delta - \frac{\theta}{\psi} [\mu + x_t + \sigma \varepsilon_{t+1}] + \theta [(-\log \kappa_1 + \mu) +  (\kappa_1 A_x \rho - A_x + 1) x_t  + \kappa_1 A_x \varphi_e \sigma e_{t+1} + \sigma \varepsilon_{t+1}]\\
&= 
\Bigg[\theta \log \delta 
- \frac{\theta}{\psi} \mu 
+  \theta (-\log \kappa_1 + \mu) \Bigg]
+  \Bigg[\theta(\kappa_1 A_x \rho - A_x + 1) - \frac{\theta}{\psi}\Bigg] x_t  
+ \theta\Bigg[1 - \frac{1}{\psi}  \Bigg]\sigma\varepsilon_{t+1} 
+ \theta\kappa_1 A_x \varphi_e \sigma e_{t+1}
\end{align*}

The expected value and variance of $m_{t+1} + r_{c,t+1}$ is

\begin{align*}
E_t[m_{t+1} + r_{c,t+1}] 
&= \Bigg[\theta \log \delta 
- \frac{\theta}{\psi} \mu 
+  \theta (-\log \kappa_1 + \mu) \Bigg]
+  \Bigg[\theta(\kappa_1 A_x \rho - A_x + 1) - \frac{\theta}{\psi}\Bigg] x_t \\
Var_t[m_{t+1} + r_{c,t+1}]
&= \theta^2 \Bigg[1 - \frac{1}{\psi}  \Bigg]^2\sigma^2 
+ \theta^2 \kappa_1^2 A_x^2 \varphi_e^2 \sigma^2
\end{align*}

Thus, we can plug the expected value and variance back in:

\begin{align*}
0&= E_t[m_{t+1} + r_{i,t+1}] + \frac{1}{2} Var_t[m_{t+1} + r_{i,t+1}]\\
\implies
0 &= \Bigg[\theta \log \delta 
- \frac{\theta}{\psi} \mu 
+  \theta (-\log \kappa_1 + \mu) \Bigg]
+  \Bigg[\theta(\kappa_1 A_x \rho - A_x + 1) - \frac{\theta}{\psi}\Bigg] x_t 
+ \frac{1}{2} \theta^2 \Bigg[1 - \frac{1}{\psi}  \Bigg]^2\sigma^2 
+ \frac{1}{2} \theta^2 \kappa_1^2 A_x^2 \varphi_e^2 \sigma^2 \\
\implies &
\begin{cases}
0 &=\Bigg[\theta \log \delta 
- \frac{\theta}{\psi} \mu 
+  \theta (-\log \kappa_1 + \mu) \Bigg]
+ \frac{1}{2} \theta^2 \Bigg[1 - \frac{1}{\psi}  \Bigg]^2\sigma^2 
+ \frac{1}{2} \theta^2 \kappa_1^2 A_x^2 \varphi_e^2 \sigma^2 \\
0 &=  \Bigg[\theta(\kappa_1 A_x \rho - A_x + 1) - \frac{\theta}{\psi}\Bigg]
\end{cases}\\
\implies
A_x &= \frac{1 - 1/\psi}{1 - \kappa_1 \rho}
\end{align*}


Thus, asset valuations respond positive to expected growth if 

\begin{align*}
A_x > 0
\iff 
\frac{1 - 1/\psi}{1 - \kappa_1 \rho} > 0 
\iff
1 > 1/\psi 
\iff
\psi > 1
\end{align*}

Economically, this parameter restriction means that the substitution effect dominates the wealth effect.

Using the solution for $A_x$, we can express $r_{c,t+1}$ as 

\begin{align*}
r_{c,t+1} &= [-\log \kappa_1 + \mu] +  [\kappa_1 A_x \rho - A_x + 1] x_t  + \kappa_1 A_x \varphi_e \sigma e_{t+1} + \sigma \varepsilon_{t+1}\\
&= [-\log \kappa_1 + \mu] +  \frac{1}{\psi} x_t  + \frac{(1-1/\psi)\kappa_1 \varphi_e }{1- \kappa_1 \rho} \sigma e_{t+1} + \sigma \varepsilon_{t+1}\\
&= r_{c,0} +  \frac{1}{\psi} x_t  + B_x \varphi_e\sigma e_{t+1} + B_c \sigma \varepsilon_{t+1}
\end{align*}

where $r_{c,0} \equiv -\log \kappa_1 + \mu$, $B_x \equiv \frac{(1-1/\psi)\kappa_1 }{1- \kappa_1 \rho}$, and $B_c \equiv 1$.

\pagebreak

\item Solve for $M_{t+1}$.

\bigskip

\textbf{Solution:} 

\begin{align*}
m_{t+1} 
&= \theta \log \delta - \frac{\theta}{\psi} \Delta c_{t+1} + (\theta - 1)r_{c,t+1}\\
&= \theta \log \delta - \frac{\theta}{\psi} \Bigg[\mu + x_t + \sigma \varepsilon_{t+1} \Bigg] + (\theta - 1)\Bigg[r_{c,0} +  \frac{1}{\psi} x_t  + B_x \varphi_e \sigma e_{t+1} + B_c \sigma \varepsilon_{t+1}\Bigg]\\
&= \Bigg[\theta \log \delta - \frac{\theta}{\psi} \mu + (\theta - 1) r_{c,0}\Bigg]
+ \Bigg[-\frac{\theta}{\psi} + \frac{\theta - 1}{\psi} \Bigg] x_t
+ (\theta - 1)B_x \varphi_e \sigma e_{t+1}
+ (-\theta/\psi + (\theta - 1)B_c) \sigma \varepsilon_{t+1}\\
&= m_0 - m_x x_t -\lambda_x \varphi_e \sigma e_{t+1} - \lambda_c \sigma \varepsilon_{t+1}
\end{align*}

where

\begin{align*}
m_0 &\equiv \Bigg[\theta \log \delta - \frac{\theta}{\psi} \mu + (\theta - 1) r_{c,0}\Bigg]\\
m_x &\equiv -1/\psi \\
\lambda_x 
&\equiv (1 - \theta) B_x\\
&= (1 - \theta )\frac{(1-1/\psi)\kappa_1 }{1- \kappa_1 \rho}\\
\lambda_c 
&\equiv \theta/\psi - \theta + 1 \\
&= - \gamma
\end{align*}

If $\psi > 1 \implies \lambda_x > 0$ and $\lambda_c < 0$ by assumption.

\bigskip

\item Consumption strip at maturity $n = 1$.

\bigskip

\textbf{Solution:}

\begin{align*}
P_{t,1} &= E_t[M_{t+1} C_{t+1}]\\
\implies
\frac{P_{t,1}}{C_t} &= E_t \Bigg[M_{t+1} \frac{C_{t+1}}{C_t} \Bigg]\\
&= E_t [\exp(m_{t+1} + \Delta c_{t+1} ) ]\\
&= \exp(E_t [m_{t+1} + \Delta c_{t+1} ] + \frac{1}{2} V_t[m_{t+1} + \Delta c_{t+1} ])\\
\implies
pc_{t,1} &= E_t [m_{t+1} + \Delta c_{t+1} ] + \frac{1}{2} V_t[m_{t+1} + \Delta c_{t+1} ] \\
\implies
m_{t+1} + \Delta c_{t+1} 
&= m_0 - m_x x_t -\lambda_x \varphi_e \sigma e_{t+1} - \lambda_c \sigma \varepsilon_{t+1} + \mu + x_t + \sigma \varepsilon_{t+1}\\
&= m_0 + \mu - (m_x -1)x_t -\lambda_x \varphi_e \sigma e_{t+1} - (\lambda_c - 1) \sigma \varepsilon_{t+1}\\
\implies
E_t [m_{t+1} + \Delta c_{t+1} ] 
&= m_0 + \mu - (m_x -1)x_t \\
\implies
V_t [m_{t+1} + \Delta c_{t+1} ]
&=\lambda_x^2 \varphi_e^2 \sigma^2 + (\lambda_c - 1)^2 \sigma^2 \\
pc_{t,1} &= m_0 + \mu + (1 - m_x)x_t + \frac{1}{2} \lambda_x^2 \varphi_e^2 \sigma^2 + \frac{1}{2}(\lambda_c - 1)^2 \sigma^2 
\end{align*}

\pagebreak

\item Solve for the return on the consumption strip and risk premium

\bigskip

\textbf{Solution:} The return on the consumption strip is:

\begin{align*}
\log r_{t+1,1} 
&= \Delta c_{t+1} - pc_{t,1} \\
&= \mu + x_t + \sigma \varepsilon_{t+1} - m_0 - \mu - (1 - m_x)x_t - \frac{1}{2} \lambda_x^2 \varphi_e^2 \sigma^2 - \frac{1}{2}(\lambda_c - 1)^2 \sigma^2\\
&=  - m_0 + m_x x_t + \sigma \varepsilon_{t+1}  - \frac{1}{2} \lambda_x^2 \varphi_e^2 \sigma^2 - \frac{1}{2}(\lambda_c - 1)^2 \sigma^2
\end{align*}

The risk premium of the consumption strip is

\begin{align*}
-Cov_t(r_{t+1, 1}, m_{t+1}) 
&= -Cov_t(- m_0 + m_x x_t + \sigma \varepsilon_{t+1}  - \frac{1}{2} \lambda_x^2 \varphi_e^2 \sigma^2 - \frac{1}{2}(\lambda_c - 1)^2 \sigma^2, \\& m_0 - m_x x_t -\lambda_x \varphi_e \sigma e_{t+1} - \lambda_c \sigma \varepsilon_{t+1})\\
&= -Cov_t( \sigma \varepsilon_{t+1}  , -\lambda_x \varphi_e \sigma e_{t+1} - \lambda_c \sigma \varepsilon_{t+1}) \\
&= \lambda_c \sigma^2 \\
&= -\gamma \sigma^2
\end{align*}

The risk premium on the consumption claim is

\begin{align*}
-Cov_t(r_{c,t+1}, m_{t+1}) 
&= -Cov_t(r_{c,0} +  \frac{1}{\psi} x_t  + B_x \varphi_e\sigma e_{t+1} + B_c \sigma \varepsilon_{t+1}, m_0 - m_x x_t -\lambda_x \varphi_e \sigma e_{t+1} - \lambda_c \sigma \varepsilon_{t+1})\\
&= -Cov_t( B_x \varphi_e\sigma e_{t+1} + B_c \sigma \varepsilon_{t+1}, -\lambda_x \varphi_e \sigma e_{t+1} - \lambda_c \sigma \varepsilon_{t+1})\\
&= B_x\lambda_x \varphi_e^2 \sigma^2 + B_c \lambda_c \sigma^2\\
&= (1-\theta)B_x^2 \varphi_e^2\sigma^2 -\gamma \sigma^2
\end{align*}

The consumption strip has a negative risk premium and the consumption claim has a higher risk premium, so this model is inconsistent with the evidence that short-term consumptions strips have higher average excess returns than claim on all future cash-flows.

\begin{align*}
-Cov_t(r_{t+1, 1}, m_{t+1}) &< -Cov_t(r_{c,t+1}, m_{t+1}) \\
\iff
-\gamma \sigma^2 &< (1-\theta)B_x^2 \varphi_e^2\sigma^2 -\gamma \sigma^2\\
\iff
0 &< (1-\theta)B_x^2 \varphi_e^2\sigma^2
\end{align*}

\item Time-varying risk premium on consumption strips.

\bigskip

\textbf{Solution:} No, this model implies a constant risk premium on consumption strips.  We can introduce time-varying volatility $\sigma_t$. With time-varying volatility, the risk premium of the consumption strip would be $-\gamma \sigma_t^2$, which would vary over time.

\end{enumerate}

\pagebreak

\section{Problem 3}


\pagebreak

\section{Problem 4}

\end{document}