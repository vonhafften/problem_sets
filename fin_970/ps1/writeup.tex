\documentclass{article}
\usepackage{amsmath,amsthm,amssymb,amsfonts}
\usepackage{setspace,enumitem}
\usepackage{graphicx}
\usepackage{hyperref}
\usepackage{natbib}
\usepackage{afterpage}
\usepackage{xcolor}
\usepackage{etoolbox}
\usepackage{booktabs}
\usepackage{pdfpages}
\usepackage{multicol}
\usepackage{geometry}
\usepackage{accents}
\usepackage{bbm}
\usepackage{placeins}
\hypersetup{
	colorlinks,
	linkcolor={blue!90!black},
	citecolor={red!90!black},
	urlcolor={blue!90!black}
}

\newtheorem{theorem}{Theorem}
\newtheorem{assumption}{Assumption}
\newtheorem{definition}{Definition}
\newtheorem{lemma}{Lemma}
\setlength{\parindent}{0cm}
\geometry{margin = 1in}

\newcommand{\R}{\mathbb{R}}
\newcommand{\ubar}[1]{\underaccent{\bar}{#1}}
\newcommand{\Int}{\text{Int}}
\newcommand{\xbf}{\mathbf{x}}
\newcommand{\Abf}{\mathbf{A}}
\newcommand{\Bbf}{\mathbf{B}}
\newcommand{\Gbf}{\mathbf{G}}
\newcommand{\bbf}{\mathbf{b}}
\newcommand{\one}{\mathbbm{1}}

\newtoggle{extended}
\settoggle{extended}{false}

\title{FIN 970: Homework 1}
\author{Alex von Hafften }

\begin{document}

\maketitle


\section{Problem 1: GMM Estimation of a Linear Regression Model}

Write a code to implement a GMM estimation of a linear regression model, $Y_t = \beta'X_t + u_t$. The code should produce the point estimates and the Newey-West standard errors of $\beta$ and the regression $R^2$. We will use this code in later assignments to evaluate statistical significance of predictability evidence

\bigskip

\textbf{Solution:} See \texttt{gmm.jl} for implementation. Also see \texttt{gmm\_test.jl} for a test of the GMM estimation using simulated data.

\section{Problem 2: Bayesian Estimation of an Autoregressive Model}

Consider an $AR(1)$ model for $y^T = \{y_t\}_{t=1}^T$:

$$
y_{t+1} = \mu + \rho y_t+ \sigma \varepsilon_{t+1}
$$

where $\varepsilon \sim_{iid} N(0, 1)$

\begin{enumerate}

\item Consider independent conjugate priors for the model parameters,

$$
\mu \sim N(m, s^2), \rho \sim N(\tilde\rho, \omega^2), \sigma^2 \sim IG(\alpha/2, \beta/2)
$$

Show that the conditional posteriors are given by

$$
\mu | y^T, \rho, \sigma \sim N, \rho | y^T, \mu, \sigma \sim N, \sigma^2 | y^T, \mu, \rho \sim IG
$$

Find the parameters of the posterior distributions in terms of the parameters of the prior and
the data.

\textbf{Solution:} The priors for the model parameters imply:

\begin{align*}
\mu &\sim N(m, s^2)\\
\implies f(\mu) 
&= \frac{1}{\sqrt{2\pi s^2}} \exp \Bigg( - \frac{1}{2s^2} (\mu - m)^2\Bigg) 
\propto \exp \Bigg( - \frac{1}{2s^2} (\mu^2 - 2\mu m) \Bigg) \\
\rho &\sim N(\tilde\rho, \omega^2) \\
\implies f(\rho) 
&= \frac{1}{\sqrt{2\pi \omega^2}} \exp \Bigg( - \frac{1}{2 \omega^2} (\rho - \tilde\rho)^2\Bigg) 
\propto \exp \Bigg( - \frac{1}{2\omega^2} (\rho^2 - 2\rho \tilde \rho) \Bigg) \\ 
\sigma^2 &\sim IG(\alpha/2, \beta/2) \\
\implies f(\sigma^2)
&= \frac{(\beta/2)^{(\alpha/2)}}{\Gamma(\alpha/2)} (\sigma^2)^{-\alpha/2 - 1} \exp\Bigg( - \frac{\beta}{2\sigma^2} \Bigg)  
\propto \sigma^{2(-\alpha/2 - 1)} \exp \Bigg( - \frac{\beta}{2\sigma^2}\Bigg)
\end{align*}

From the $AR(1)$ structure, we know that 

\begin{align*}
y_{t} | \mu, \rho, \sigma, y_{t-1} &\sim N(\mu + \rho y_{t-1}, \sigma^2) \\ 
f(y_t|y_{t-1}, \mu, \rho, \sigma) &= \frac{1}{\sqrt{2\pi \sigma^2}} \exp \Bigg( - \frac{1}{2 \sigma^2} (y_t - \mu - \rho y_{t-1})^2\Bigg) \\
&= \frac{1}{\sqrt{2\pi \sigma^2}} \exp \Bigg( - \frac{1}{2 \sigma^2} (y_t^2 + \mu^2 + \rho^2 y_{t-1}^2 - 2\mu y_t - 2 \rho y_{t-1}y_t + 2\rho \mu y_{t-1})\Bigg) \\
&\propto \frac{1}{\sigma}\exp \Bigg( - \frac{1}{2 \sigma^2} (y_t^2 + \mu^2 + \rho^2 y_{t-1}^2 - 2\mu y_t - 2 \rho y_{t-1}y_t + 2\rho \mu y_{t-1})\Bigg)
\end{align*}

Furthermore, assuming $y_0$ is given (so $f(y_0|\mu, \rho, \sigma) = 1$):

\begin{align*}
f(y^T|\mu, \rho, \sigma) 
&= f(y_T|\mu, \rho, \sigma, y_{T-1}) \cdot ... \cdot f(y_1|\mu, \rho, \sigma, y_0) f(y_0|\mu, \rho, \sigma) \\
&= \prod_{t=1}^T f(y_t |\mu, \rho, \sigma, y_{t-1})\\
&\propto \prod_{t=1}^T \frac{1}{\sigma}\exp \Bigg( - \frac{1}{2 \sigma^2} (y_t^2 + \mu^2+ \rho^2 y_{t-1}^2 - 2\mu y_t - 2 \rho y_{t-1}y_t + 2\rho \mu y_{t-1})\Bigg)\\
&= \frac{1}{\sigma^T} \exp \Bigg(\sum_{t=1}^T  - \frac{1}{2 \sigma^2} (y_t^2 + \mu^2+ \rho^2 y_{t-1}^2 - 2\mu y_t - 2 \rho y_{t-1}y_t + 2\rho \mu y_{t-1})\Bigg)\\
&= \frac{1}{\sigma^T} \exp \Bigg( - \frac{1}{2 \sigma^2} (\sum_{t=1}^T y_t^2 + T\mu^2 + \rho^2 \sum_{t=1}^T y_{t-1}^2 - 2\mu \sum_{t=1}^T y_t - 2 \rho \sum_{t=1}^T y_{t-1}y_t + 2\rho \mu \sum_{t=1}^T y_{t-1})\Bigg)\\
&= \frac{1}{\sigma^T} \exp \Bigg( - \frac{T}{2 \sigma^2} (\frac{1}{T}\sum_{t=1}^T y_t^2 + \mu^2 + \frac{\rho^2}{T} \sum_{t=1}^T y_{t-1}^2 - \frac{2\mu}{T} \sum_{t=1}^T y_t - \frac{2 \rho}{T} \sum_{t=1}^T y_{t-1}y_t + \frac{2\rho \mu }{T}\sum_{t=1}^T y_{t-1})\Bigg)\\
&= \frac{1}{\sigma^T} \exp \Bigg( - \frac{T}{2 \sigma^2} (\overline{y_T^2} + \mu^2 +  \rho^2 \overline{y_{T-1}^2} - 2\mu \overline{ y_T } - 2 \rho \overline{ z_T} + 2\rho \mu \overline{y_{T-1}})\Bigg)
\end{align*}

where

\begin{align*}
\overline{y_{T}^2} &\equiv \frac{1}{T} \sum_{t=1}^{T} y_t^2\\
\overline{y_{T-1}^2} &\equiv \frac{1}{T} \sum_{t=1}^{T} y_{t-1}^2\\
\overline{y_{T}} &\equiv \frac{1}{T} \sum_{t=1}^{T} y_{t}\\
\overline{y_{T-1}} &\equiv \frac{1}{T} \sum_{t=1}^{T} y_{t-1}\\
\overline{z_{T}} &\equiv \frac{1}{T} \sum_{t=1}^{T} y_ty_{t-1}\\
\end{align*}

\pagebreak

Applying Bayes' Rule for $\mu$, we know that:

\begin{align*}
f(\mu|y^T, \rho, \sigma) &\propto f(\mu) f(y^T|\mu, \rho, \sigma)\\
&\propto \frac{1}{\sigma^T} \exp \Bigg( - \frac{1}{2s^2} (\mu^2 - 2\mu m) \Bigg) \exp \Bigg( - \frac{T}{2 \sigma^2} (\overline{y_T^2} + \mu^2  + \rho^2 \overline{y_{T-1}^2} - 2\mu \overline{ y_T } - 2 \rho \overline{ z_T} + 2\rho \mu \overline{y_{T-1}})\Bigg)\\
&\propto \exp \Bigg( - \frac{1}{2s^2} (\mu^2 - 2\mu m) \Bigg) \exp \Bigg( - \frac{T}{2 \sigma^2} (\mu^2  + 2\rho \mu \overline{y_{T-1}}- 2\mu \overline{ y_T })\Bigg)\\
&= \exp \Bigg( - \frac{1}{2s^2} (\mu^2 - 2\mu m) - \frac{T}{2 \sigma^2} (\mu^2  + 2\rho \mu \overline{y_{T-1}}- 2\mu \overline{ y_T })\Bigg)\\
&= \exp \Bigg(- \frac{1}{2}\Bigg(\mu^2\Bigg(\frac{1}{s^2} + \frac{T}{\sigma^2} \Bigg) - 2\mu \Bigg(\frac{m}{2s^2} + \frac{T(\rho  \overline{y_{T-1}}- \overline{ y_T }))}{2 \sigma^2} \Bigg)\Bigg)\Bigg)\\
&= \exp \Bigg(- \frac{1}{2\Big(\frac{1}{s^2} + \frac{T}{\sigma^2} \Big)^{-1}}\Bigg(\mu^2 - 2\mu \Bigg(\frac{m}{2s^2} + \frac{T(\rho  \overline{y_{T-1}}- \overline{ y_T }))}{2 \sigma^2} \Bigg)\Bigg(\frac{1}{s^2} + \frac{T}{\sigma^2} \Bigg)^{-1}\Bigg)\Bigg)
\end{align*}

Thus, $\mu| y^T, \mu, \rho, \sigma \sim N(\tilde{m}, \tilde{s}^2)$ where:

\begin{align*}
\nu_\mu
&\equiv \frac{\sigma^2}{s^2}\\
\tilde{s}^2 
&\equiv \Bigg(\frac{1}{s^2} + \frac{T}{\sigma^2} \Bigg)^{-1} \\
&= \Bigg(\frac{\nu_\mu}{\sigma^2} + \frac{T}{\sigma^2} \Bigg)^{-1} \\
&= \frac{\sigma^2}{\nu_\mu+T} \\
\tilde{m} 
&\equiv \Bigg(\frac{m}{2s^2} + \frac{T(\rho  \overline{y_{T-1}}- \overline{ y_T }))}{2 \sigma^2} \Bigg)\Bigg(\frac{1}{s^2} + \frac{T}{\sigma^2} \Bigg)^{-1}\\
&= \Bigg(\frac{m\nu_\mu + T(\rho  \overline{y_{T-1}}- \overline{ y_T }))}{2 \sigma^2} \Bigg) \frac{\sigma^2}{\nu_\mu+T}\\
&= m \frac{\nu_\mu}{\nu_\mu + T} + (\rho  \overline{y_{T-1}}- \overline{ y_T })\frac{T}{\nu_\mu + t}
\end{align*}

\pagebreak

Applying Bayes' Rule for $\rho$, we know that:

\begin{align*}
f(\rho|y^T, \sigma, \mu) &\propto f(\rho) f(y^T|\mu, \rho, \sigma)\\
&\propto \frac{1}{\sigma^T} \exp \Bigg( - \frac{1}{2\omega^2} (\rho^2 - 2\rho \tilde \rho) \Bigg) \exp \Bigg( - \frac{T}{2 \sigma^2} (\overline{y_T^2} + \mu^2  + \rho^2 \overline{y_{T-1}^2} - 2\mu \overline{ y_T } - 2 \rho \overline{ z_T} + 2\rho \mu \overline{y_{T-1}})\Bigg)\\
&\propto \exp \Bigg( - \frac{1}{2\omega^2} (\rho^2 - 2\rho \tilde \rho) - \frac{T}{2 \sigma^2} (\rho^2 \overline{y_{T-1}^2} - 2 \rho \overline{ z_T} + 2\rho \mu \overline{y_{T-1}})\Bigg) \\
&= \exp \Bigg(-\frac{1}{2}  \Bigg(\rho^2 \Bigg(\frac{1}{\omega^2} + \frac{T\overline{y_{T-1}^2}}{\sigma^2}\Bigg)  +  2\rho\Bigg( \frac{\tilde\rho}{\omega^2} + \frac{T\overline{ z_T}}{\sigma^2} - \frac{T\mu\overline{y_{T-1}}}{\sigma^2} \Bigg)\Bigg)\Bigg)\\
&= \exp \Bigg(-\frac{1}{2\Big(\frac{1}{\omega^2} + \frac{T\overline{y_{T-1}^2}}{\sigma^2}\Big)^{-1}}  \Bigg(\rho^2   +  2\rho\Bigg( \frac{\tilde\rho}{\omega^2} + \frac{T\overline{ z_T}}{\sigma^2} - \frac{T\mu\overline{y_{T-1}}}{\sigma^2} \Bigg)\Bigg(\frac{1}{\omega^2} + \frac{T\overline{y_{T-1}^2}}{\sigma^2}\Bigg)^{-1}\Bigg)\Bigg)
\end{align*}

Thus, $\rho | y^T, \mu, \rho, \sigma \sim N(\tilde{\tilde{\rho}}, \tilde{\omega}^2)$ where:

\begin{align*}
\nu_\rho
&\equiv \frac{\sigma^2}{\omega^2}\\
\tilde{\omega}^2
&\equiv \Bigg(\frac{1}{\omega^2} + \frac{T\overline{y_{T-1}^2}}{\sigma^2}\Bigg)^{-1}\\
&= \Bigg(\frac{\nu_\rho}{\sigma^2} + \frac{T\overline{y_{T-1}^2}}{\sigma^2}\Bigg)^{-1}\\
&= \frac{\sigma^2}{\nu_\rho + T\overline{y_{T-1}^2}}\\
\tilde{\tilde{\rho}}
&= \Bigg( \frac{\tilde\rho\nu_\rho}{\sigma^2} + \frac{T\overline{ z_T}}{\sigma^2} - \frac{T\mu\overline{y_{T-1}}}{\sigma^2} \Bigg)  \frac{\sigma^2}{\nu_\rho + T\overline{y_{T-1}^2}}\\
&=  \frac{\nu_\rho}{\nu_\rho + T\overline{y_{T-1}^2}}\tilde\rho + \frac{T}{\nu_\rho + T\overline{y_{T-1}^2}} [\overline{ z_T} - \mu\overline{y_{T-1}}]
\end{align*}

Applying Bayes' Rule for $\sigma$, we know that:

\begin{align*}
f(\sigma|y^T, \rho, \mu) 
&\propto f(\rho) f(y^T|\mu, \rho, \sigma)\\
&\propto \frac{1}{\sigma^T} \sigma^{2(-\alpha/2 - 1)} \exp \Bigg( - \frac{\beta}{2\sigma^2}\Bigg) \exp \Bigg( - \frac{T}{2 \sigma^2} (\overline{y_T^2} + \mu^2 +  \rho^2 \overline{y_{T-1}^2} - 2\mu \overline{ y_T } - 2 \rho \overline{ z_T} + 2\rho \mu \overline{y_{T-1}})\Bigg)\\
&= \sigma^{2(-\alpha/2 - 1) - T} \exp \Bigg( - \frac{\beta}{2\sigma^2} - \frac{T}{2 \sigma^2} (\overline{y_T^2} + \mu^2 +  \rho^2 \overline{y_{T-1}^2} - 2\mu \overline{ y_T } - 2 \rho \overline{ z_T} + 2\rho \mu \overline{y_{T-1}})\Bigg)\\
&= \sigma^{2(-(\alpha+T)/2 - 1)} \exp \Bigg( - \frac{1}{2\sigma^2} [\beta + T (\overline{y_T^2} + \mu^2 +  \rho^2 \overline{y_{T-1}^2} - 2\mu \overline{ y_T } - 2 \rho \overline{ z_T} + 2\rho \mu \overline{y_{T-1}})]\Bigg)\\
\end{align*}

\pagebreak

Thus, $\sigma | y^T, \mu, \rho, \sigma \sim IG(\tilde{\alpha}/2, \tilde{\beta}/2)$ where:

\begin{align*}
\tilde{\alpha} &= \alpha + T\\
\tilde{\beta} &= \beta + T (\overline{y_T^2} + \mu^2 +  \rho^2 \overline{y_{T-1}^2} - 2\mu \overline{ y_T } - 2 \rho \overline{ z_T} + 2\rho \mu \overline{y_{T-1}})
\end{align*}


\item The Excel file \texttt{Longyielddata.xls} contains the annual data for long-term 10-year US
government bond yields from Global Financial Data (GFD) database.

Choose the priors for the model parameters. For example, we can use fairly uninformative
priors:

\begin{center}
\begin{tabular}{ l | c c }
& Prior Mean & Prior Std. Dev. \\ 
\hline
 $\mu$  & 0.3 & 0.5  \\  
 $\rho$ & 0.95 & 0.2 \\
 $\sigma^2$ & 1 & 1
\end{tabular}
\end{center}

Design and implement an MCMC algorithm to draw a long sample from the joint posterior
distribution of the three parameters, where the sampling of each parameter is implemented
via Gibbs sampler.

...

\item Run a long MCMC chain and discard appropriate number of initial draws due to burn-in.
Plot the three chains. Do the chains mix well? Does it look like they come from a stationary
distribution? How large is the persistence in the chains? Plot the autocorrelation plots for
the chains, and the scatter plots of each parameter against the others. You might find that
the most problematic is the persistence and cross-correlation between $\mu$ and $\rho$, and there's
a good reason for it. Notice that � is an intercept, and not the unconditional mean of $y_t:E(y) = \mu/(1 - \rho)$. So every time we change $\rho$, intuitively, the draw for $\mu$ has to adjust to
target the unconditional mean of the process. How would you change the problem to break
up this almost mechanical correlation between the parameters?

...

\item Report the posterior means and standard deviations of the parameters - how different the
posterior means are from the prior, and from the standard OLS estimates in the data? How
do the prior and posterior standard deviations compare? Overall, do you think your results
look reasonable?

...

\item  If you carefully examine your chain for $\rho$, you might find that some draws are above 1. This
does not look reasonable, as we have good reasons to believe that interest rates are stationary.
Let's correct that. First, let's use a truncated Normal prior for $\rho$ to ensure that this parameter
is always in $(-1, 1)$:

$$
f(\rho) \propto N (\tilde\rho, \omega^2) \text{ if } \rho \in (-1, 1); f(\rho) = 0 \text{ otherwise}.
$$

Further, instead of using Gibbs to draw $\rho$, now let's use Independence Metropolis-Hastings
where the proposal density is equal to the Normal conditional posterior of $\rho$ you found in part
1). Design and implement the new MCMC algorithm. Notice that a new algorithm (known
as a Rejection Sampling) should be a very simple and intuitive modification of the previous
one.

...

\item Examine the chain, report the posterior means and standard deviations. Overall, do you think your results look reasonable?

...

\item Having simulation output from the MCMC chains makes it easy to compute, and assess
significance of, any complicated non-linear functions of the parameters and the data. Suppose
we are interested in $N = 5$ year forecast of yields from the model. Show that the forecast is
given by,

$$
E_ty_{t+N} = \mu \frac{1 - \rho^N}{1 - \rho} + \rho^N y_t
$$

Fix $t$. You can compute the implied forecast $(E_t y_{t+N})^i$ for each parameter draw from the chain $\theta^i$, for $i$ from 1 to $M$. Having the distribution of time-$t$ forecasts $\{(E_t y_{t+N})^i\}_{i=1}^M$, you
can numerically compute their mean, and 2.5\% - 97.5\% confidence band. Now do it for all $t$, and plot the posterior mean and the confidence band for the yield forecasts from the model.

...

\end{enumerate}

\pagebreak

\section{Problem 3: Latent Drift Model}

Consider the following specification for consumption dynamics:

\begin{align*}
\Delta c_{t+1} &= \mu + x_t + \sigma_c \eta_{t+1},\\
x_{t+1} &= \rho x_t + \sigma_x e_{t+1},
\end{align*}

where $\eta$ and $e$ are independent (over time and from each other) shocks with mean zero and variance one.

\begin{enumerate}

\item We are interested in estimating the parameters of the model: $\mu, \sigma_c, \rho, \sigma_x$. Consider the following four moments: $E(\Delta c_t), Var(\Delta c_t), Cov(\Delta c_t, \Delta c_{t-1}), Cov(\Delta c_t, \Delta c_{t-2})$. Show that these four moments exactly identify the four unknown parameters. Describe how you would design and implement a GMM estimation of the model parameters based on these four moments. How would you infer the unobserved $x_t$ based on these estimates?

\bigskip

\textbf{Solutions:} We can rewrite $x_t$:

\begin{align*}
x_{t}
&= \rho x_{t-1} + \sigma_x e_{t} \\
&= \rho (\rho x_{t-2} + \sigma_x e_{t-1}) + \sigma_x e_{t} \\
&= \rho^2x_{t-2} + \sigma_x (\rho e_{t-1} + e_t)\\
&= \rho^2(\rho x_{t-3} + \sigma_x e_{t-2}) + \sigma_x (\rho e_{t-1} + e_t)\\
&= \rho^3 x_{t-3} + \sigma_x (\rho^2 e_{t-2} + \rho e_{t-1} + e_t)\\
&= \rho^t x_0 + \sigma_x \sum_{j=0}^{t-1} \rho^j e_{t-j}\\
&= \sigma_x \sum_{j=0}^{t-1} \rho^j e_{t-j}
\end{align*}

assuming $x_0 = 0$.  Thus, 

\begin{align*}
E[x_t] 
&= \sigma_x \sum_{j=0}^{t-1} \rho^j E[e_{t-j}] \\
&= 0\\
E[x_t^2] 
&= \sigma_x^2 \sum_{j=0}^{t-1} \rho^{2j} E[e_{t-j}^2]\\
&= t\sigma_x^2 \rho^{2j}\\
Var[x_t] 
&= E[x_t^2] - [E[x_t]]^2 \\
&= t\sigma_x^2 \rho^{2j}
\end{align*} 

\begin{align*}
E[\Delta c_t] 
&= E[\mu + x_{t-1} + \sigma_c \eta_t]\\
&= \mu + E[x_{t-1}] + \sigma_c E[\eta_t]\\
&= \mu
\end{align*}

\begin{align*}
E[(\Delta c_t)^2] 
&= E[(\mu + x_{t-1} + \sigma_c \eta_t)(\mu + x_{t-1} + \sigma_c \eta_t)]\\
&= E[\mu^2 + x_{t-1}^2 + \sigma_c^2 \eta_t]\\
&= \mu^2 + (t-1)\sigma_x^2 \rho^{2j}
\end{align*}

\begin{align*}
Var[\Delta c_t] 
&= E[(\Delta c_t)^2]  - [E[\Delta c_t] ]^2]\\ 
&= (t-1)\sigma_x^2 \rho^{2j}
\end{align*}

\begin{align*}
Cov(\Delta c_t, \Delta c_{t-1}) 
&= Cov(\mu + x_{t-1} + \sigma_c \eta_{t}, \mu + x_{t-2} + \sigma_c \eta_{t-1}) \\
&= Cov(\mu + (\rho x_{t-2} + \sigma_x e_{t-1}) + \sigma_c \eta_{t}, \mu + x_{t-2} + \sigma_c \eta_{t-1}) \\
&= \rho Cov( x_{t-2} , x_{t-2}) \\
&= \rho (t-2) \sigma_x^2 \rho^{2j}
\end{align*}

\begin{align*}
Cov(\Delta c_t, \Delta c_{t-2})
&= Cov(\mu + x_{t-1} + \sigma_c \eta_{t}, \mu + x_{t-3} + \sigma_c \eta_{t-2}) \\
&= Cov(\mu + (\rho x_{t-2} + \sigma_x e_{t-1}) + \sigma_c \eta_{t}, \mu + x_{t-3} + \sigma_c \eta_{t-2}) \\
&= Cov(\mu + (\rho (\rho x_{t-3} + \sigma_x e_{t-2}) + \sigma_x e_{t-1}) + \sigma_c \eta_{t}, \mu + x_{t-3} + \sigma_c \eta_{t-2}) \\
&= \rho^2 Cov( x_{t-3} , x_{t-3}) \\
&= \rho^2 \rho (t-3) \sigma_x^2 \rho^{2j} \\
\end{align*}

\item  The shocks $\eta$ and $e$ are assumed to be independent from each other. Can we estimate the correlation between the shocks in the data? If so, show what data moments would identify it.

...

\end{enumerate}

\end{document}